\section{Derivatives of Containers}\label{derivatives}

Let us recall the notion of a container in type theory:
\begin{definition}
  A \emph{container} \( (\MkCont{S}{P}) \) consists of \emph{shapes} \( S : \Type \) and
  a family \( P : S \to \Type \) of \emph{positions}.
  We access shapes and positions via postfix projections
  \( (\MkCont{S}{P})_{\Sh} \DefEq S \) and \( (\MkCont{S}{P})_{\Ps} \DefEq P \).
\end{definition}

Containers were introduced to model inductive data types,
hence they are closed under products \( \times \), sums \( + \) and substitution \( \Subst{\Blank}{\Blank} \);
see \autoref{container-operations} for their definitions.
The constant container at a type \( S \) is \( \CConst{S} \DefEq (\MkCont{S}{0}) \);
products and sums form a monoidal structure with units \( \CConst{1} \) and \( \CConst{0} \).
The identity container \( \Id \DefEq (\MkCont{1}{1}) \)
is a unit for substitution, which is a non-symmetric monoidal product.

\begin{figure}
  \begin{align*}
    % Products
    (F \CTimes G)_{\Sh} &\DefEq S \times T
      &
    (F \CTimes G)_{\Ps} &\DefEq \lambda (s, t).\, P_s + Q_t
      \\
    % Sums
    (F \CPlus G)_{\Sh} &\DefEq S + T
      &
    (F \CPlus G)_{\Ps} &\DefEq
      \lambda
      \begin{cases}
        \Inl(s).\, P_s \\
        \Inr(t).\, Q_t
      \end{cases}
      \\
    % Substitution
    (\Subst{F}{G})_{\Sh} &\DefEq \sum_{s : S} (P_s \to T)
      &
    (\Subst{F}{G})_{\Ps} &\DefEq \lambda (s, f).\, \sum_{p : P_s} Q_{fp}
  \end{align*}
  \caption{%
    Operations on containers \( F \JudgeEq (\MkCont{S}{P}) \) and \( G \JudgeEq (\MkCont{T}{Q}) \).
  }%
  \label{container-operations}
\end{figure}

A shape \( s : S \) of a container \( F \JudgeEq (\MkCont{S}{P}) \) is intuitively understood
as the name of a constructor of the inductive type encoded by \( F \),
and each position \( p : P(s) \) indexes an argument to this constructor.
To model (polymorphic) functions between inductive types,
one should describe how constructors of the input map to constructors in the output,
and associate to each position in the output --- encoding the argument to a constructor --- its occurrence in the input.
Hence, a morphism \( (\MkCont{S}{P}) \to (\MkCont{T}{Q}) \) consists of a map of shapes \( f : S \to T \),
and a family of maps of positions \( u : \prod_{s : S} Q_{fs} \to P_s \).
For our purposes, we have to go one step further and describe a kind of \enquote{linear} function;
one in which positions are mapped 1-to-1 between input and output:
\begin{definition}
  Let \( F \JudgeEq (\MkCont{S}{P}) \) and \( G \JudgeEq (\MkCont{T}{Q}) \).
  The type of \emph{cartesian morphisms} between \( F \) and \( G \) is
  \[
    \Cart{F}{G} \DefEq \sum_{f : S \to T} \prod_{s : S} Q_{fs} \simeq P_s
  \]
  We denote the shape- and position components of a morphism \( {f : \Cart{F}{G}} \)
  by \( {f_{\Sh} : F_{\Sh} \to G_{\Sh}} \) and \( f_{\Ps} : \prod_{s : F_{\Sh}} G_{\Ps}(f_{\Sh}(s)) \simeq F_{\Ps}(s) \), respectively.
\end{definition}

In the remainder of this paper we will only consider cartesian morphisms.
We are going to drop \enquote{cartesian} in writing, but retain the notation \( \Cart{F}{G} \).
Morphisms of containers compose as expected, and together with an identity morphism \( \Op{id}_F : \Cart{F}{F} \)
they form a wild category \( \ContCart \):
composition is associative and unital, but we make no assumption on the truncation level of the hom-types \( \Cart{F}{G} \).
Note that this wild category is \emph{not} univalent in the na{\"i}ve sense:
The canonical map taking paths of containers \( F = G \) to categorical isomorphisms,\footnote{%
  that is, pairs \( f : \Cart{F}{G}, g : \Cart{G}{F} \) with chosen paths \( fg = \Op{id}_F \) and \( gf = \Op{id}_G \)
}
is \emph{not} an equivalence, unless shapes and positions of the involved containers are sets.
Instead, we are going to use the following definition when comparing containers:

\begin{definition}
  A cartesian morphism \( (f, u) : \Cart{F}{G} \) is an \emph{equivalence} of containers
  if \( f : F_{\Sh} \to G_{\Sh} \) is an equivalence of types.
  We write \( F \CartEquiv G \) for the type of equivalences of containers.
\end{definition}
By an application of univalence, the type of equivalences \( F \CartEquiv G \) is equivalent to the type of paths, \( F = G \).
While we cannot prove it internally, we think of the wild category of containers as an \( (\infty,1) \)-category
in which \( F \CartEquiv G \) is the \enquote{correct} notion of weak equivalence,
representing the \( \infty \)-groupoid of paths \( F = G \).

In cases where we do care about the truncation level of shapes and positions,
we define the following subtypes of containers:
\begin{definition}
  A container \( (\MkCont{S}{P}) \) is \emph{\( (n,k) \)-truncated} if \( S \) is \( n \)-truncated,
  and \( P_s \) is \( k \)-truncated for all \( s : S \).
  Write \( \ContCart_{n,k} \) for the wild subcategory of \( (n,k) \)-truncated containers.
  A container is \emph{discrete} if \( P_s \) is a discrete type for all \( s : S \).
\end{definition}
Traditional set-based containers are \( (0,0) \)-truncated.
In particular, \( \ContCart_{0,0} \) forms a univalent 1-category in which a morphism being an isomorphism is a proposition equivalent to it being an equivalence of containers.
An example of containers of higher truncation level are \citeauthor{Gylterud2011}'s \emph{symmetric containers}~\cite{Gylterud2011}:
these have groupoids for shapes, and sets for positions, hence are exactly the \( (1,0) \)-truncated containers.

When constructing a morphism \( f : \Cart{F}{G} \), we will oftentimes factor it
through (equivalent) auxiliary containers
\begin{equation*}
  \begin{tikzcd}
    F \ar[r, -multimap, "f"] & G \\
    {F\mathrlap{'}} \ar[r, -multimap, "f'"{swap}] & {G\mathrlap{'}}
    \ar[from=1-1,to=2-1, multimap-multimap]
    \ar[from=2-2,to=1-2, multimap-multimap]
  \end{tikzcd}
\end{equation*}
This lets us separate the bureaucracy of bringing \( F \) and \( G \) into a comparable shape
from the act of defining an interesting morphism \( f' : \Cart{F'}{G'} \).
In particular, \( f \) is an equivalence of containers if and only if \( f' \) is,
which is often easier to characterize.

Similarly, we will often implicitly use the following extensionality principle
to construct paths between container morphisms:
\begin{lemma}[note={Extensionality of container morphisms}]\label{container-morphism-ext}
  Let \( F, G : \Cont \) and \( f, g : \Cart{F}{G} \).
  The type \( f = g \) is equivalent to
  \[
    \adjustlimits\prod_{s : F_{\Sh}} \sum_{p : f_{\Sh}(s) = g_{\Sh}(s)} f_{\Ps}(s) =_p^{B} g_{\Ps}(s)
  \]
  in which the dependent path varies over the family \( B(t) \DefEq G_{\Ps}(t) \to F_{\Ps}(s) \).
  \begin{proof}
    By extensionality for paths in \( \Sigma \)- and \( \Pi \)-types, and the fact that paths of equivalences are the same as paths of their underlying functions.
  \end{proof}
\end{lemma}

\subsection{Derivatives, Universally}

The derivative of a container \( G \) represents a type of
\( G \)-shaped trees in which a chosen subtree has been removed.
For traditional containers, this can be implemented as an operation \( G \mapsto \Der{G} \) by removing a chosen position over each shape:
Given \( G \JudgeEq (\MkCont{T}{Q}) \), \citeauthor{AbbottEtAl2005DataDifferentiating} define \( \Der{G} \)
to have as shapes pairs \( (t , q) : \sum_{T} Q \), over which the positions are \( Q_t \setminus q \).
To ensure that \( \Der \) is well-behaved, it is characterized by a universal property:
on a suitable subcategory of containers,\footnote{namely that of discrete containers} \( \Der \) extends to an endofunctor,
and one shows the existence of an adjunction \( \Maybe{\Blank} \dashv \Der \)
that describes morphisms into \( \Der{G} \):
If such an adjunction exists, then morphisms \( \Cart{F}{\Der{G}} \) are in 1-to-1 correspondence with morphisms of shape \( (f, u) : \Cart{\Maybe{F}}{G} \).
On positions, such morphisms are equivalences \( u_s : G_{\Ps}(fs) \simeq F_{\Ps}(s) + 1 \),
i.e.\@ maps that \enquote{avoid} the removed position \( u_s^{\Inv}(\Inr(\bullet)) : G_{\Ps}(fs) \).

As we have seen in the previous section, removing points from a type is a subtle process in a univalent setting:
a position \( q : G_{\Ps}(t) \) is not simply a discrete point, but comes with a potentially complicated type of paths around it.
If we wanted to encode morphisms into \( \Der{G} \) by the same universal property,
we would have to avoid the entire connected component around \( q \),
that is, find some type of \enquote{hole} \( H(q) \) such that \( G_{\Ps}(fs) \simeq F_{\Ps}(s) + H(q) \).
But this type now depends on \( q \) and its higher path structure,
and can no longer be expressed uniformly as a simple product with the fixed container \( \Id \).

From this, we can devise two ways forward:
Either we characterize the derivative in terms of a more fine-grained universal property
that takes the dependency on higher paths into account,
or we only take derivatives with respect to positions whose path types are more uniform.
For the purpose of this paper we take the second approach,
and define a derivative in terms of \emph{isolated} positions:
\begin{definition}
  The derivative of a container \( \Der{(\MkCont{S}{P})} \DefEq (\MkCont{S'}{P'}) \)
  has shapes \( S' \DefEq \sum_{s : S} \Isolated{P_s} \) and positions \( P'(s , p) \DefEq P_s \setminus p \).
\end{definition}

We can not only take the derivative of a container, but also act functorially on morphisms:
\begin{problem}
  Define a wild endofunctor \( \Der : \ContCart \to \ContCart \).
  That is,
  for all \( f : \Cart{F}{G} \), a morphisms \( \Der{f} : \Cart{\Der{F}}{\Der{G}} \),
  such that \( \Der(\Op{id}_F) = \Op{id}_{\Der{F}} \) and \(\Der(fg) = (\Der{f})(\Der{g}) \).
  \begin{construction}
    For any \( (f, u) : \Cart{(\MkCont{S}{P})}{(\MkCont{T}{Q})} \),
    there is a canonical morphism \( \Der{(f, u)} \DefEq (f' , u') : \Cart{\Der{(\MkCont{S}{P})}}{\Der{(\MkCont{T}{Q})}} \)
    obtained as follows:
    On shapes, the map \( f' : \sum_{s : S} \Isolated{P_s} \to \sum_{t : T} \Isolated{Q_t} \)
    applies \( f \) to the first component and \( \Isolated{(u_s^\Inv)} : \Isolated{P_s} \simeq \Isolated{Q_{fs}} \) to the second.
    On positions, \( u'_{s,p} : G_{fs} \setminus u_s^\Inv(p) \simeq F_s \setminus p \) is obtained from \( u_s \), which respects the removed point \( p \).
    (cf.\@ \autoref{is-isolated-respect-equiv}).
  \end{construction}
\end{problem}

Since isolated points always form a set, taking the derivative of a container preserves its truncation level
as long as shapes are at least sets, and positions are at least propositions:
\begin{proposition}
  For \( n \geq 0 \) and \( k \geq -1 \), the derivative of an \( (n, k) \)-truncated container is \( (n, k) \)-truncated.
  \begin{proof}
    Let \( (\MkCont{S}{P}) \) an \( (n, k) \)-truncated container.
    By \autoref{is-set-isolated} \( \Isolated{P_s} \) is a 0-truncated type and \( S \) is \( n \)-truncated,
    thus \( \Der{(\MkCont{S}{P})}_{\Sh} \JudgeEq \sum_{s : S} \Isolated{P_s} \) is \( n \)-truncated.
    Positions are \( k \)-types since \( P_s \setminus p \) embeds into \( P_s \).
  \end{proof}
\end{proposition}

Importantly, this turns \( \Der \) into an endofunctor on the 1-category of set-truncated containers,
without having to assume that the containers are discrete:
\begin{corollary}
  Taking derivatives is an endofunctor \( \Der : \ContCart_{0,0} \to \ContCart_{0,0} \).
\end{corollary}
We believe that analouges of this hold for higher truncation levels.
Symmetric containers, for example, form a bicategory \( \ContCart_{1,0} \),
and it should be straightforward (albeit tedious) to show that \( \Der \) restricts to a pseudo\-functor on this bicategory.

We now show that this generalized derivative operation is right-adjoint to \( \Maybe{\Blank} \),
verifying that it has the desired universal property.
We define this adjunction in terms of unit- and counit natural transformations,
and discuss how this relates to the original construction of \citeauthor{AbbottEtAl2005DataDifferentiating}.

\begin{problem}\label{derivative-adjunction}
  Define a wild adjunction \( (\eta, \varepsilon) : \Maybe{\Blank} \dashv \Der \),
  that is:
  \begin{enumerate}
    \item
      Two families of morphisms
      \[
        \eta : \prod_{F : \Cont} \Cart{F}{\Der{(F \CTimes \Id)}}
        \quad\text{and}\quad
        \varepsilon : \prod_{G : \Cont} \Cart{\Der{G} \CTimes \Id}{G}
      \]
    \item with fillers of naturality squares
      \begin{align*}
        &
          \begin{tikzcd}[ampersand replacement=\&]
            F
              \ar[r, -multimap, "\eta_F"]
              \ar[d, -multimap, "f"{swap}]
              \&
            \Der(\Maybe{F})
              \ar[d, -multimap, "\Der(\Maybe{f})"]
              \\
            G
              \ar[r, -multimap, "\eta_G"{swap}]
              \&
            \Der(\Maybe{G})
          \end{tikzcd}
        &
          &\text{and}
        &
        &
          \begin{tikzcd}[ampersand replacement=\&]
            \Maybe{\Der{F}}
              \ar[r, -multimap, "\varepsilon_F"]
              \ar[d, -multimap, "\Maybe{\Der{f}}"{swap}]
              \&
            F
              \ar[d, -multimap, "f"]
              \\
            \Maybe{\Der{G}}
              \ar[r, -multimap, "\varepsilon_G"{swap}]
              \&
            G
          \end{tikzcd}%
      \end{align*}
      for all \( f : \Cart{F}{G} \),
    \item and zigzag-diagrams
      \begin{align*}
        &
        \begin{tikzcd}[ampersand replacement=\&, column sep=tiny, row sep=large]
          \Maybe{F} \& \& \Maybe{F} \\
                    \& \Maybe{(\Der{(\Maybe{F})})} \& %
          \ar[from=1-1, to=1-3, -multimap, "\Op{id}"]
          \ar[from=1-1, to=2-2, -multimap, "\Maybe{{\eta_F}}"{'}]
          \ar[from=2-2, to=1-3, -multimap, "\varepsilon_{\Maybe{F}}"{'}]
        \end{tikzcd}
        &
          &\text{and}
        &
        &
        \begin{tikzcd}[ampersand replacement=\&, column sep=tiny, row sep=large]
                    \& \Der{(\Maybe{\Der{G}})} \& \\
          \Der{G} \& \& \Der{G} %
          \ar[from=2-1, to=2-3, -multimap, "\Op{id}"]
          \ar[from=2-1, to=1-2, -multimap, "\eta_{\Der{G}}"]
          \ar[from=1-2, to=2-3, -multimap, "\Der{(\varepsilon_{G})}"]
        \end{tikzcd}
      \end{align*}
  \end{enumerate}
  \begin{construction}
    Let \( F \JudgeEq (\MkCont{S}{P}) \) and define \( \eta_F : \Cart{F}{\Der{(\Maybe{F})}} \).
    On shapes, \( {\eta_F^{\Sh}} : S \to \sum_{(s, \Blank) : S \times 1} \Isolated{(P_s + 1)} \)
    sends \( s \) to \( (s, \bullet) \) and \( \mathsf{nothing} \);
    the latter is isolated by \autoref{is-isolated-nothing}.
    On positions, define \( \eta_F^{\Ps} : \prod_{s : S} (P_s + 1) \setminus \Nothing \simeq P_s \)
    as in \autoref{maybe-minus-nothing-equiv}.

    Let \( G \JudgeEq (\MkCont{T}{Q}) \); define the counit \( \varepsilon_G : \Cart{\Maybe{(\Der{G})}}{G} \) as follows:
    on shapes, \( \varepsilon_G^{\Sh} : \sum_{t : T} \Isolated{Q_t} \times 1 \to T \) is the first projection.
    On positions the equivalence
    \(
      \varepsilon_G^{\Ps}(t , q) : Q_t \simeq (Q_t \setminus q) + 1
    \)
    is given by \autoref{isolated-minus-plus-equiv} for all \( t : T \) and \( q : \Isolated{Q_t} \).

    To construct the zigzag-fillers,
    we apply the necessary extensionality principles for functions, equivalences and sum types.
    We are left to construct paths that are almost \( \Op{refl} \);
    only some proofs of isolation and removal need to be compared up to propositional equality.
    Construction of the naturality squares for \( \eta \) and \( \varepsilon \) is done similarly.
  \end{construction}
\end{problem}

In their original construction, \citeauthor{AbbottEtAl2005DataDifferentiating} only establish isomorphisms between hom-sets
\( \Cart{\Maybe{F}}{G} \) and \( \Cart{F}{\Der{G}} \), natural in \( F \).
This falls short of defining a proper adjunction since \( \Der{G} \) is left undefined for non-discrete containers \( G \).
We can however complete the search for a suitable subcategory of differentiable containers \cite[14]{AbbottEtAl2005DataDifferentiating}:
Our derivative is defined functorially for \emph{all} containers,
and restricting the above wild adjunction to set-truncated containers yields the following:
\begin{theorem}\label{derivative-adjunction-sets}
  In the 1-category of set-truncated containers \( \ContCart_{0,0} \), \( \Der \) is right-adjoint to tensoring \( \Blank \CTimes \Id \). \qed
\end{theorem}
From this, we can extract the familiar natural isomorphism of hom-sets in \( \ContCart_{0,0} \).
In fact, the same argument lets us obtain a natural equivalence of hom-types for arbitrary containers,
which otherwise would be somewhat tedious to establish:
there is an equivalence \( (\Cart{F}{\Der{G}}) \simeq (\Cart{\Maybe{F}}{G}) \) natural in \( F, G : \ContCart \),
with underlying map
\begin{align*}
  {\Blank}^\sharp &: (\Cart{F}{\Der{G}}) \to (\Cart{\Maybe{F}}{G}) \\
  f^{\sharp} &\DefEq \varepsilon_G \circ (\Maybe{f})
\end{align*}
Interestingly, the proof of \cite[{Theorem 5.1}]{AbbottEtAl2005DataDifferentiating} already uses \( {\Blank}^{\sharp} \) to show naturality of the hom-set isomorphism in \( F \).

Furthermore, we can iterate the adjunction and easily obtain a notion of \( n \)-fold derivatives.
Denote by \( \Op{y}(A) \DefEq (\MkCont{1}{A}) \) the container of \enquote{\( A \)-tuples}.
We see that \( \Der^n(G) \) encodes a type of \( G \)-terms with \( n \) holes:
\begin{corollary}
  For all \( n : \mathbb{N} \), there is an adjunction \( \Blank \CTimes \Op{y}[n] \dashv \Der^n \)
  in the 1-category \( \ContCart_{0,0} \).
\end{corollary}
We believe that this extends to the entire wild category \( \ContCart \),
but one has to carefully check that the data of the wild adjunction in \autoref{derivative-adjunction} composes coherently.

\subsection{Basic Laws of Derivates}

Derivatives of containers earn their name by observing how they interact with other operations on containers:
derivatives of constants are zero, derivatives of sums and products follow the familiar sum- and product rules.
Let us now investigate to which extent our derivative still respects these basic laws.

It is easy to see that derivatives of constants are always zero,
and that \( \Der{\Id} \) is the constant \( \CConst(1) \).
Both factor through the following observation:
\begin{proposition}\label{derivative-prop-trunc}
  Let \( S : \Type \) and \( P : S \to \Op{Prop} \).
  There is an equivalence of containers
  \[
    { \Der{(\MkCont{S}{P})} }
      \CartEquiv
    { (\MkCont{{\textstyle \sum_{S} P }}{0}) }
  \]
  In particular, we have
  \( \Der{(\Id)} \CartEquiv \CConst(1) \)
  and \( \Der(\CConst(A)) \CartEquiv \CConst(0) \) for all \( A : \Type \).
  \begin{proof}
    Since \( P_s \) is a proposition, we know that \( \Isolated{P_s} \simeq P_s \) and \( P_s \setminus p \simeq 0 \).
    Thus,
    \begin{align*}
      \Der{(\MkCont{S}{P})}
        &\CartEquiv (\MkCont{((s, p) : \textstyle\sum_{s : S} \Isolated{P_s})}{P_s \setminus p})
          \\
        &\CartEquiv (\MkCont{((s, p) : \textstyle\sum_{s : S} P_s)}{0})
          \qedhere
    \end{align*}
  \end{proof}
\end{proposition}

Similarly, we convince ourselves that \( \Der \) distributes over (binary) sums,
and that derivatives of products follow a Leibniz rule:
\begin{proposition}\label{sum-product-rule}\label{sum-rule}\label{leibniz-rule}
  For containers \( F, G \), the following hold:
  \begin{enumerate}
    \item Sum rule: \( {\Der{(F \CPlus G)}} \CartEquiv {\Der{F} \CPlus \Der{G}} \)
    \item Leibniz rule: \( {\Der{(F \CTimes G)}} \CartEquiv {(\Der{F} \CTimes G) \CPlus (F \CTimes \Der{G})} \)
  \end{enumerate}
  \begin{proof}
    Let \( F \JudgeEq (\MkCont{S}{P}) \) and \( G \JudgeEq (\MkCont{T}{Q}) \).
    Both equivalences are established like in the discrete setting
    (cf.~\cite[{Proposition 6.3 and 6.4}]{AbbottEtAl2005DataDifferentiating}),
    with one exception:
    to derive the Leibniz rule,
    one needs to show that isolated points distribute over binary sums in
    \[
      \sum_{s : S} \sum_{t : T} \Isolated{(P_s + Q_t)}
        \simeq
      \sum_{s : S} \sum_{t : T} \Isolated{P_s} + \Isolated{Q_t},
    \]
    which is done via \autoref{isolated-sum-equiv}.
  \end{proof}
\end{proposition}

To some extent we can also solve differential equations involving \( \Der \).
In particular, given a container \( F \), we can ask if it has an anti-derivative,
i.e. some \( G \) for which \( \CartIso{\Der{G}}{F} \).
Interestingly, \( \Der \) has fixed-points, that is containers that are their own derivative.
The prototypical example of such a fixed-point is the container of finite multisets, or \emph{bags}.
Famously, bags cannot be expressed as set-truncated containers, but can be if one allows shapes of higher truncation level.
Let here reproduce \citeauthor{Gylterud2011}'s construction~\cite[{example~3.6.1}]{Gylterud2011} of bags as a symmetric containers (with a groupoid of shapes),
but in the language of HoTT.
Recall that a type is considered finite
if there is some \( n : \mathbb{N} \) for which it is merely equivalent to \( \Fin{n} : \Type \).%
\footnote{Specifically, such types are called Bishop-finite~\cite[{Definition~4.4}]{FruminEtAl2018FiniteSetsHomotopy}.}
The universe of finite sets,
\( \FinSet \DefEq \sum_{X : \Type} \sum_{n : \mathbb{N}} \lVert X \simeq \Fin{n} \rVert \),
comes with a map \( \El \DefEq \Op{fst} : \FinSet \to \Type \) projecting out the underlying type.
Together, these form the container of \emph{bags}, \( \Op{Bag} \DefEq (\MkCont{\FinSet}{\El}) \).
As written, the shapes of this container quantify over all types, hence live in a higher universe.
There are however equivalent small replacements of this type,
such as the one given by \citeauthor{FinsterEtAl2021CartesianBicategoryPolynomial} in \cite[Theorem~25]{FinsterEtAl2021CartesianBicategoryPolynomial}.

\begin{proposition}\label{bag-fixed-point}
  The bag-container is a fixed-point of derivation:
  there is an equivalence \( \CartIso{\Der{\Op{Bag}}}{\Op{Bag}} \).
  \begin{proof}
    First, note that finite sets are closed under addition and removal of points:
    if \( X \) is finite, then so are \( X + 1 \) and \( X \setminus x \) for all \( x : X \).
    On shapes, we construct an equivalence
    \(
      \sum_{X : \FinSet} \Isolated{\El(X)} \simeq \FinSet
    \)
    from mutually inverse functions \( f \) and \( g \).
    From left to right, define \( f(X, x_0) \DefEq X \setminus x_0 \);
    the other way let \( g(X) \DefEq (X + 1 , \Nothing) \).
    By univalence, finite sets are equal if their carrier types are equivalent,
    so \( f \) and \( g \) are inverses by \autoref{isolated-minus-plus-equiv} and \autoref{maybe-minus-nothing-equiv}.
    Given \( X : \FinSet \) and \( x_0 : \El(X) \), positions are related by the identity equivalence,
    that is
    \(
      \Op{Bag}_{\Ps}(f(X, x_0)) = \El(f(X, x_0)) = X \setminus x_0 = \Der{\Op{Bag}}_{\Ps}(X, x_0)
    \).
  \end{proof}
\end{proposition}

Unlike in classical analysis, where the exponential function is the unique solution to
the differential equation \( f^{\prime} = f \) with initial condition \( f(0) = 1 \),
the situation for containers is more nuanced:
While \( \Op{Bag} \) is a solution for \( \CartIso{\Der{F}}{F} \) such that \( { F[\CConst{0}] \CartEquiv \CConst{1} } \),
it is far from being the only one.
This is not entirely unexpected:
containers are closely related to Joyal's \emph{combinatorial species}
(as discussed e.g.~in Yorgey's thesis \cite[67]{Yorgey2014CombinatorialSpeciesLabelled}),
and these are known to have many non-isomorphic solutions even for simple differential equations,
as shown by Labelle in~\cite{Labelle1986combinatorialdifferentialequations}.

Modulo size issues, the proof of \autoref{bag-fixed-point} goes through for any subuniverse
of types closed under addition and removal of single points:
\begin{proposition}
  Let \( P : \Type \to \Op{Prop} \) a predicate
  for which for all \( A : \Type \), \( P(A) \) implies both \( P(A + 1) \)
  and \( \prod_{a : A} P(A \setminus a) \).
  This defines a container
  \( \Op{Bag}_P \DefEq (\MkCont{\sum_{\Type} P}{\Op{fst}}) \)
  such that \( \CartIso{\Der{\Op{Bag}_P}}{\Op{Bag}_P} \). \qedhere
\end{proposition}
For the details of the proof we refer the reader to the formalization;
there one can find an example of this applied to countable multisets
defined by the predicate \( P(A) \DefEq \lVert A \hookrightarrow \mathbb{N} \rVert \).
