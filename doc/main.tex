\documentclass{article}

\usepackage{amsmath}
\usepackage{amssymb}
\usepackage[backend=biber,style=alphabetic]{biblatex}
\usepackage{mathtools}
\usepackage{bussproofs}
\usepackage{cmll}
\usepackage{csquotes}
\usepackage{enumitem}
\usepackage{tikz-cd}
\usepackage{keytheorems}
\usepackage{todonotes}
\usepackage{xcolor}
\usepackage{hyperref}

\hypersetup{
  colorlinks=true,
  % linkcolor=cyan,
}

\NewDocumentCommand{\TypeTopology}{m O{}}{\href{https://www.cs.bham.ac.uk/~mhe/TypeTopology/#1.html\IfValueT{#2}{\##2}}{\texttt{#1}}}

\newkeytheoremstyle{boxed}{
  tcolorbox-no-titlebar={sharp corners=all,boxrule=1pt,colback=black!8},
}
\newkeytheoremstyle{boxed-definition}{
  inherit-style=boxed,
  % qed=$\lrcorner$,
  numbered=false,
}
\newkeytheoremstyle{marked-definition}{
  inherit-style=definition,
  qed=$\lrcorner$,
}
\newkeytheoremstyle{construction}{
  inherit-style=definition,
  headfont={\itshape},
  qed=$\lrcorner$,
  numbered=false,
}
\newkeytheoremstyle{todo}{
  tcolorbox-no-titlebar={sharp corners=all,boxrule=1pt,colback=orange!10},
  numbered=false,
  headfont={\scshape},
}
\newkeytheorem{todo-block}[style=todo,name={Todo}]
\newkeytheorem{theorem}[style=plain,parent=section]
\newkeytheorem{
  proposition,
  lemma,
  corollary,
  problem,
  conjecture,
}[
  style=plain,
  sibling=theorem,
]
\newkeytheorem{definition}[style=marked-definition,sibling=theorem]
\newkeytheorem{construction}[style=construction,sibling=theorem]
\newkeytheorem{remark}[style=remark,sibling=theorem]
\newkeytheorem{example}[style=example,sibling=theorem]

% A custom tikzcd arrow tip, a lollipop (⊸).
% https://tex.stackexchange.com/a/730357
\tikzset{
  % See section 4.4 ("Glyph arrow tips") of the tikzcd manual.
  % multimap/.tip={Glyph[glyph math command=circ, glyph length=1.03ex, glyph shorten=-0.2ex]},
  multimap/.tip={Circle[open]},
}

\colorlet{Highlight}{purple!80!gray}
\NewDocumentCommand{\Highlight}{m}{\textcolor{Highlight}{#1}}

\title{Derivates of Containers without Decidable Equality}
\author{Philipp Joram}
\date{\today}

\addbibresource{bibliography.bib}

% https://damaru2.github.io/general/notations_with_links/
% \NewDocumentCommand{\newlink}{m m}{\protect\hyperlink{#1}{#2}}


\DeclareMathSymbol{\shortminus}{\mathbin}{AMSa}{"39}

\newcommand*{\Op}[1]{\mathsf{#1}}
\DeclareMathOperator{\Type}{\mathsf{Type}}
\DeclareMathOperator{\W}{\mathsf{W}}
\DeclareMathOperator{\FinSet}{\Op{FinSet}}
\DeclareMathOperator{\El}{\Op{El}}
\DeclareMathOperator{\Sup}{\mathsf{sup}}
\DeclareMathOperator{\Dec}{\mathsf{Dec}}
\DeclareMathOperator{\DefEq}{\mathrel{\coloneq}}
\DeclareMathOperator{\JudgeEq}{\mathrel{\doteq}}
\NewDocumentCommand{\Inv}{}{{\shortminus 1}}
% \NewDocumentCommand{\Blank}{}{\shortminus}
\def\blankskip{0mu minus 1fill}
\NewDocumentCommand{\Blank}{}{{\mskip\blankskip\mathunderscore\mskip\blankskip}}

\DeclareMathOperator{\Isolate}{\Op{isolate}}
\DeclareMathOperator{\Replace}{\Op{replace}}
\DeclareMathOperator{\SigmaRemove}{\Op{\Sigma-remove}}
\DeclareMathOperator{\SigmaIsolate}{\Op{\Sigma-isolate}}

\newcommand*{\Isolated}[1]{#1^{\circ}}
\DeclareMathOperator{\Graft}{\Op{graft}}
\DeclarePairedDelimiterXPP{\GraftSyntaxX}[3]%
  {}%
  {[}%
  {]}%
  {_{#3}}%
  {#2 \delimsize\vert #1}
\DeclarePairedDelimiterX{\GraftSyntax}[2]{[}{]}{#2 \delimsize\vert #1}

\newcommand*{\MkCont}[2]{#1\mathbin{\triangleleft}#2}
\DeclareMathOperator{\Sh}{\mathsf{Sh}}
\DeclareMathOperator{\Ps}{\mathsf{Ps}}

\NewDocumentCommand{\CTimes}{}{\mathbin{\times}}
\NewDocumentCommand{\CPlus}{}{\mathbin{+}}
\NewDocumentCommand{\CConst}{}{\Op{k}}
\DeclarePairedDelimiterXPP{\Wk}[1]%
  {} % pre-code
  {\langle} % left
  {\rangle} % right
  {^{\uparrow}} % post-code
  {#1} % body
\newcommand*{\Subst}[2]{#1[#2]}

\NewDocumentCommand{\SimMultiMap}{}{%
  \mathrel{%
    \vbox{%
      \offinterlineskip%
      \mathsurround=0pt
      \ialign{%
        \hfil##\hfil\cr
        \normalfont\scalebox{0.9}{\kern-.33ex{$\sim$}}\cr
        \noalign{\kern-.25ex}
        $\scriptstyle\multimap$\cr%
      }%
    }%
  }%
}

\newcommand*{\Cart}[2]{#1\mathrel{\multimap}#2}
\NewDocumentCommand{\CartEquiv}{}{\multimapboth}
\newcommand*{\CartIso}[2]{#1 \mathrel{\CartEquiv} #2}

\DeclareMathOperator{\Cont}{\mathsf{Cont}}
\NewDocumentCommand{\ContCart}{}{\operatorname{\Cont}^{{\scriptscriptstyle\multimap}}}

\NewDocumentCommand{\Der}{}{\partial}
\NewDocumentCommand{\Id}{}{\mathsf{Id}}
\NewDocumentCommand{\Proj}{m}{\pi_{#1}}
\DeclareMathOperator{\Chain}{\Op{chain}}
\DeclareMathOperator{\In}{\Op{in}}
\DeclareMathOperator{\MuRule}{\Op{\mu-rule}}

\DeclareMathOperator{\Inl}{\mathsf{inl}}
\DeclareMathOperator{\Inr}{\mathsf{inr}}

\NewDocumentCommand{\Maybe}{m}{#1 \CTimes \Id}
\DeclareMathOperator{\Nothing}{\mathsf{nothing}}
\DeclareMathOperator{\Just}{\mathsf{just}}

\makeatletter
\newcommand{\proofsubparagraph}{%
  \@startsection{subparagraph}%
  {5}%
  {\z@}%
  {3.25ex \@plus1ex \@minus .2ex}%
  {-1em}%
  {\sffamily\normalsize\bfseries}%
}
\makeatother

\begin{document}

\maketitle

\begin{abstract}
  We introduce a derivative operation for containers whose positions lack decidable equality.
\end{abstract}

\section{Introduction}

A container \cite{AbbottEtAl2005ContainersConstructingstrictly} is a concise encoding
of an inductive data type:
it consists of a collection of shapes \( S \), each \( s : S \) representing a constructor,
and for each a collection of positions \( P(s) \), indexing arguments to the constructor.
Many operations on data types can be made precise as operations on containers,
including sums \( F \CPlus G \), products \( F \CTimes G \) and substitution \( \Subst{F}{G} \), allowing us to reason about them.
\Citeauthor{Huet1997Zipper} presents \citetitle{Huet1997Zipper}~\cite{Huet1997Zipper},
an informal procedure for computing a type of contexts around a subtree in a tree-like data type.
In~\cite{McBride2001DerivativeRegularType}, \Citeauthor{McBride2001DerivativeRegularType} turns this into an algorithm for the inductively defined class of \enquote{regular types},
and notices that this type of one-hole contexts behaves like a derivative with respect to constants,
sums and products, and that substitution of types follows a chain rule.
This is extended to containers by \citeauthor{AbbottEtAl2003DerivativesContainers},
first in a categorical meta-language~\cite{AbbottEtAl2003DerivativesContainers},
then in a type-theoretic one~\cite{AbbottEtAl2005DataDifferentiating}.
They show that derivatives of containers satisfy a universal property with respect to a class of \enquote{linear}
morphisms of containers,
namely that for any container \( G \) on which equality of positions is decidable,
linear morphisms \( F \multimap \Der{G} \) are in 1-to-1 correspondence with morphisms \( F \CTimes \Id \multimap G \).
Furthermore, the laws of derivatives are encoded as (linear) isomorphisms:
the chain rule, for example, becomes an isomorphism of containers
\( \Der{(\Subst{F}{G})} \cong \Subst{(\Der{F})}{G} \CTimes \Der{G} \).

In the meantime, other notions of containers have appeared,
whose shapes and positions have different kinds of structure:
quotient- \cite{AbbottEtAl2004ConstructingPolymorphicPrograms} and action containers~\cite{JoramVeltri2025DataTypesSymmetries}
express symmetries by having groups act on sets of positions,
symmetric containers internalize symmetries in a groupoid of shapes~\cite{Gylterud2011},
and generalized containers~\cite{AltenkirchKaposi2021containermodeltype} describe directed relations of shapes as a category.

When working in Univalent Foundations, the most direct generalization of containers is that from sets to arbitrary types
with potentially non-trivial higher path types.
What can we say about containers whose shapes and positions are such untruncated types?
Much of the theory transfers directly, since it never assumed that types were sets in the first place.
Even smallest-~\cite{DamatoEtAl2025FormalisingInductiveCoinductive} and largest fixed points~\cite{AhrensEtAl2015NonWellfoundedTrees} can be proved to exist.

In this paper, we show that derivatives can be defined for untruncated containers,
and that they satisfy an analogue of the universal property of discrete containers.
This is mostly an exercise in reverse mathematics:
\emph{What are the minimal assumptions on decidability that we can make such that a derivative is still well-defined?}
To this end, we recall the notion of \emph{isolatedness} of a point in a type, and show how it interacts with various type formers.
Being isolated is a local notion of discreteness, in the sense that around an isolated point, a type is discrete.
Our derivative ranges only over isolated positions;
this allows us to prove the following universal property
by application of simple lemmata for isolated points:
On the (wild) category \( \ContCart \) of containers and cartesian morphisms,
\( \Der \) is an endofunctor that is right-adjoint to taking products, \( \Maybe{\Blank} \).
As a consequence, we can solve a problem left open in \cite[14]{AbbottEtAl2005DataDifferentiating}:
the wild adjunction restricts to an ordinary adjunction on the 1-category of set-truncated containers.
In particular, this proves that \emph{every} set-truncated container has a well-behaved derivative, not just discrete ones.
Our exercise paid off: by generalizing to arbitrary types, we learn something about sets.

Verifying that this generalized derivative satisfies the expected basic laws is straightforward;
we essentially repurpose the proofs for the special case of discrete containers,
but distribute isolated points over type formers whenever necessary.
This works to show that \( \Der \) distributes over sums and products,
but fails for the chain rule:
it generalizes only to a lax chain rule, that is a cartesian morphism
from
\( \Subst{(\Der{F})}{G} \CTimes \Der{G} \)
to
\( \Der{(\Subst{F}{G})} \).

\begin{todo-block}
  Continue here.
\end{todo-block}

\begin{itemize}
  \item
    Two perspectives:
      \begin{itemize}
        \item \emph{(Reverse mathematics)}.
          What are the minimal assumptions on decidability that we can make such
          that a derivative is still well-defined?
        \item \emph{(Topology)}.
          If we think of types as spaces, which points can we remove in a \emph{continuous} manner?
      \end{itemize}
    \item
      Recall notion of isolated points
      \begin{itemize}
        \item not a new notion (\cite{EscardocontributorsTypeTopology}, \cite{KrausEtAl2013GeneralizationsHedberg’sTheorem})
        \item used to define derivatives of arbitrary containers
        \item laws of containers reduce to properties of isolated points
        \item
          Later on (\autoref{lax-chain-rule}),
          it will become necessary to investigate whether isolated points distribute over other type formers,
          in particular dependent sums.
      \end{itemize}
\end{itemize}

\subsection*{Overview}

\begin{enumerate}
  \item Removing points:
    \begin{itemize}
      \item define isolated points: points around which a type is locally trivial
      \item define removal of points, show that it works well for isolated points
      \item define \enquote{grafting} that for functions out of types with a chosen isolated point
    \end{itemize}
  \item Introduce derivatives of untruncated containers
    \begin{itemize}
      \item recall that traditionally, derivatives of containers require discrete shapes,
        or else they won't satisfy a good universal property.
      \item
        define derivatives for all container, but only involving isolated positions
      \item
        yields a (wild) adjunction for arbitrary containers, like in the set-truncated case
      \item
        derivative satisfies the same basic properties with respect to constants,
        sums and products
    \end{itemize}
  \item The chain rule is more involved
    \begin{itemize}
      \item in general, only one direction works, we get a lax chain rule
      \item the lax chain rule is always an embedding
      \item strong iff isolated points distribute over \( \Sigma \)-types
      \item a strong chain rule for arbitrary types is inconsistent; assuming it leads to a contradiction
      \item a strong chain rule for arbitrary sets means that all sets are discrete
    \end{itemize}
\end{enumerate}

\subsection*{Contributions}


\subsection*{Notation}

\begin{itemize}
  \item
    Type formers: unit type \( 1 \),
    empty type \( 0 \),
    products \( A \times B \),
    coproducts \( A + B \),
    function types \( A \to B \)
  \item Dependent type formers:
    dependent function types \( \Pi_{A} B \) or \( \prod_{a : A} B(a) \),
    dependent sum types \( \Sigma_{A} B \) or \( \sum_{a : A} B(A) \)
  \item \( \W \)-types: mention, but introduce later when needed
  \item Map over \( \Sigma \)-types with
    \[
      \Sigma(\Blank, \Blank) : \prod_{f : A \to A'} (\prod_{a : A} B(a) \to B'(f(a))) \to \sum_{A} B \to \sum_{A'} B'
    \]
  \item
    Follow the HoTT-book for transports/substitutions:
    For \( B : A \to \Type \) and \( p : x =_A y \), write \( p_* : B(x) \to B(y) \).
    Not \( \Op{subst} \) or anything like that
  \item
    Truncation levels 
  \item
    Define embeddings by two equivalent properties (invertible \( \Op{cong} \), propositional fibers, propositional fibers over the image)
  \item
    Surjections are maps with merely inhabited fibers
  \item
    Embeddings and surjections form a factorization system for equivalences
  \item equivalences satisfy a 3-for-2 property,
    we apply the insights of \cite{Jong2025FormalizingEquivalencesTears} to minimizes the tears shed over such proofs.
  \item Recall notions of wild category and functor
  \item Use \enquote{filler} for paths of morphism in a wild category
    (since in general, there can be many distinct paths)
  \item wild functor categories are \( [\mathcal{C}, \mathcal{D}] \).
\end{itemize}


\section{Isolated Points and How to Remove Them}

Recall that a type is \emph{discrete} if equality of any pair of its points is decidable.
In a univalent world, this is a rather strong property --- by Hedberg's Theorem, any such type forms a (homotopy) set,
and can thus not have any interesting path structure.
In some cases a type might have this property locally, however.
Consider for example the sum \( A + 1 \) for an arbitrary type \( A \).
Even if equality in this type is not decidable without further assumptions on \( A \),
it should at least be possible to decide \( x = \Inr(\bullet) \), no matter which \( x : A + 1 \) we are given.
After all, the constructors \( \Inl \) and \( \Inr \) are injective,
so case-analysis should yield a decision procedure, even without looking at points of \( A \).
We call such points \emph{isolated}:
\begin{definition}
  A point \( a : A \) is \emph{isolated} if \( a = b \) is decidable for all \( b : A \).
  We denote by \( \Isolated{A} \) the subtype of isolated points, that is
  \(
    \Isolated{A} \DefEq \sum_{a : A} \prod_{b : B} \Dec(a = b)
  \).
\end{definition}
\begin{remark}[note={Where does this go?}]
  In \cite[\TypeTopology{Factorial.Law}]{EscardocontributorsTypeTopology}, \citeauthor{EscardocontributorsTypeTopology}
  prove a \enquote{factorial law} for arbitrary types: if \( X! \) denotes the type of autormorphisms \( X \simeq X \),
  then \( (X + 1)! \simeq \Isolated{(X + 1)} \times X! \).
\end{remark}
Note that, \emph{a priori}, the type of proofs that some point is isolated might not be a proposition.
Take for example the circle \( S^1 \): there are \( \mathbb{Z} \)-many proofs of \( \Op{Dec}(\Op{base} =_{S^1} \Op{base}) \).\footnote{In fact \( S^1 \) is \emph{perfect}, i.e.\@ has no isolated points at all.}
However, since equality has to be decidable uniformly for a point to be isolated, this does not happen:
isolated points \emph{do} have trivial path spaces.
To prove this, \citeauthor{KrausEtAl2017NotionsAnonymousExistence} give the following
\enquote{local} version of Hedberg's Theorem:\todo{\cite{Kraus2015TruncationLevelsHomotopy} has a nice discussion in \S 3.2.}
\begin{lemma}[note={\cite[Theorem~3.12]{KrausEtAl2017NotionsAnonymousExistence}}]\label{is-prop-isolated-path}
  If \( a : A \) is isolated, then \( a = b \) is a proposition for all \( b : A \).
\end{lemma}
This has a number of important consequences:
\begin{corollary}\label{is-prop-isolated-dec-path}\label{is-prop-is-isolated}
  If \( a : A \) is isolated, then \( \Dec{(a = b)} \) is a proposition for all \( b : A \).
  In particular, being isolated is a proposition.
\end{corollary}

Taking all of this together, we see that the subtype of isolated points carves out a discrete part of a type:
\begin{proposition}[note={Isolated points form a set}]\label{is-set-isolated}
  For any type \( A \), its type of isolated points \( \Isolated{A} \) is discrete, hence a set.
  \begin{proof}
    Given \( (a, h_a), (b , h_b) : \Isolated{A} \), it suffices to show that
    \( \sum_{p : a = b} h_a =_{p} h_b \) is a proposition.
    By \autoref{is-prop-isolated-path}, \( a = b \) is a proposition.
    By \autoref{is-prop-is-isolated}, both \( h_a \) and \( h_b \) are propositions,
    hence is the dependent path type \( h_a =_{p} h_b \).
  \end{proof}
\end{proposition}


% A type whose points are all isolated is always a (homotopy) set.
% This corresponds to the intuition that, classically,
% a space consisting entirely of isolated points is necessarily (topologically) discrete.
% In a univalent setting, however, types can have interesting, non-trivial path types.
% The subtype of isolated points then carves out the topologically discrete part of a type:

As expected, being an isolated point is stable under equivalence:
\begin{lemma}\label{is-isolated-respect-equiv}
  If \( e : A \simeq B \),
  then \( a : A \) is isolated if and only if \( e(a) : B \) is isolated.
  We write \( \Isolated{e} : \Isolated{A} \simeq \Isolated{B} \) for the induced equivalence.
\end{lemma}
In cases where a map \( f : A \to B \) is not an equivalence,
we might deduce its behavior on isolated points given that it behaves \enquote{nicely} on path spaces.
Recall that \( f \) is an embedding if \( \Op{cong}_f : x = y \to f(x) = f(y) \) is an equivalence,
written \( f : A \hookrightarrow B \).
Such maps reflect isolated points:
\begin{proposition}[note={Embeddings reflect isolated points}]\label{embedding-reflect-isolated}
  Let \( f : A \hookrightarrow B \) and \( a : A \).
  If \( f(a) \) is isolated in \( B \), then \( a \) is isolated in \( A \).
  \begin{proof}
    For all \( a^\prime : A \), we need to decide \( a = a^\prime \).
    By assumption, we can decide whether \( f(a) = f(a^\prime) \) or not.
    If \( f(a) \neq f(a^\prime) \), then necessarily \( a \neq a^\prime \).
    If \( f(a) = f(a^\prime) \), we get \( a = a^\prime \) since \( f \) is an embedding, which we can cancel.
  \end{proof}
\end{proposition}
In principle, we can weaken the assumptions of the previous proposition to a function \( f : A \to B \)
for which \emph{some} \( \prod_{x,y} f(x) = f(y) \to x = y \) exists -- not necessarily an inverse to \( \Op{cong}_f \):
the proof works no matter which path we pick,
and \emph{post hoc} this choice of path is unique, since it originates from an isolated point.

In the remainder, however, we will, apply \autoref{embedding-reflect-isolated} exclusively to embeddings.
In particular, we deduce that the canonical embeddings \( \Inl : A \hookrightarrow A + B \) and \( \Inr : B \hookrightarrow A + B \)
both reflect and create isolated points:

\begin{proposition}\label{sum-embeddings-respect-isolated}
  Let \( A, B : \Type \).
  A point \( a : A \) is isolated if and only if \( \Inl{(a)} : A + B \) is isolated;
  similarly for \( b : B \) and \( \Inr{(b)} : A + B \).
  \begin{proof}
    Let \( a : A \). In the forward direction, assume that \( a \) is isolated.
    We need to show that for any \( x : A + B \), the type \( \Inl{a} = x \) is decidable.
    Consider the case of \( x \JudgeEq \Inl{a^{\prime}} \).
    We know that \( \Inl \) is an embedding, and as such there is an equivalence
    of path spaces \( (\Inl{a} = \Inl{a^\prime}) \simeq (a = a^\prime) \).
    But \( (a = a^\prime) \) is decidable by assumption,
    hence \( (\Inl{a} = \Inl{a^\prime}) \) is decidable.
    In case \( x \JudgeEq \Inr{b} \), the type \( \Inl{a} = \Inr{b} \) is empty, hence decidable.
    For the converse, apply \autoref{embedding-reflect-isolated}:
    The map \( \Inl \) is an embedding, and as such reflects isolated points.
  \end{proof}
\end{proposition}

We see that isolated points distribute over sums:
\begin{problem}\label{isolated-sum-equiv}
  Construct an equivalence \( \Isolated{(A + B)} \simeq \Isolated{A} + \Isolated{B} \).
  \begin{construction}
    Define the obvious forward- and backward maps by case analysis,
    and prove that points are isolated using \autoref{sum-embeddings-respect-isolated}.
    That these maps are mutually inverse follows since being isolated is a proposition (\autoref{is-prop-is-isolated}).
  \end{construction}
\end{problem}

From this we immediately see that \( \Nothing \DefEq \Inr{(\bullet)} : A + 1 \) is an isolated point, since \( \bullet : 1 \)
is trivially isolated:
\begin{corollary}\label{is-isolated-nothing}
  The point \( \Nothing : A + 1 \) is isolated for any type \( A \),
  and there is an equivalence \( \Isolated{(A + 1)} \simeq \Isolated{A} + 1 \).
\end{corollary}
This yields the decision procedure for \( \prod_{x : A + 1} \Op{Dec}(x = \Nothing) \) that we alluded to earlier.

While it may seem obvious that the isolated points of a disjoint sum are a sum of isolated points,
describing the isolated points of \( \Sigma \)-types is a more subtle affair.
First, observe that any dependent pair of isolated points defines an isolated point in the corresponding \( \Sigma \)-type:
\begin{proposition}\label{is-isolated-pair}\label{sigma-isolate}
  Let \( A : \Type \) and \( B : A \to \Type \) with points \( a_0 : A \) and \( b_0 : B(a_0) \).
  If both \( a_0 \) and \( b_0 \) are isolated, then \( (a_0 , b_0) \) is isolated in \( \sum_{a : A} B(a) \).
  This defines a map
  \[
    \SigmaIsolate_{A,B} :
      \sum\nolimits_{a_0 : \Isolated{A}} \Isolated{B(a_0)}
        \to
      \Isolated{%
        \big(
          \sum\nolimits_{a : A} B(a)
        \big)
      }.
  \]
  \begin{proof}
    Let \( a : A \), \( b : B(a) \). Our goal is to decide whether \( (a_0 , b_0) = (a, b) \) or not.
    By extensionality of path types of dependent sums
    it suffices to decide the equivalent type \( \sum_{p : a_0 = a} b_0 = \Op{subst}_B(p^{\Inv}, b_0) \).
    If \( a_0 \neq a \), then this type is empty.
    Otherwise, we have some \( p : a_0 = a \), and the type is inhabited or empty
    depending on whether \( b_0 = \Op{subst}_B(p^{\Inv}, b_0) \) or not.
  \end{proof}
\end{proposition}

If we had an inverse to \( \SigmaIsolate \), then the converse would hold as well.
It turns out that this requirement is not only sufficient, but also necessary:

\begin{lemma}\label{is-equiv-sigma-isolate-iff-isolated-pair}
  Let \( A : \Type \) and \( B : A \to \Type \).
  The following are equivalent:
  \begin{enumerate}
    \item \label{is-equiv-sigma-isolate-iff-isolated-pair-is-equiv}
      \( \SigmaIsolate_{A,B} \) is an equivalence.
    \item \label{is-equiv-sigma-isolate-iff-isolated-pair-pair}
      For all \( a_0 : A \) and \( b_0 : B(a_0) \),
      if \( (a_0 , b_0) \) is isolated in \( \Sigma_{A} B \),
      then both \( a_0 \) and \( b_0 \) are isolated.
  \end{enumerate}
  \begin{proof}
    The first implies the second as follows:
    Assume that \( \SigmaIsolate_{A,B} \) is an equivalence with inverse
    \( u : \Isolated{\big(\sum_{a : A} B(a)\big)} \to \sum_{a : \Isolated{A}} \Isolated{B(a)} \).
    Let \( a_0 : A \), \( b_0 : B(a_0) \), and \( h_0 : \Op{isIsolated}{(a_0, b_0)} \).
    Apply \( u \) to get some \( y \DefEq u((a_0 , b_0) , h_0) \)
    with components \( y \JudgeEq ((a, h_a), (b, h_b)) \).
    Note that \( h_a \) and \( h_b \) are proofs that \( a \) and \( b \) are isolated, respectively.
    We are done if we can show that \( p : a_0 = a \) and \( \cramped{b_0 =_{p} b} \),
    since transport along these paths preserves isolated points.
    Indeed, both \( \SigmaIsolate(y) \JudgeEq ((a, b), \mathunderscore) \) and \( ((a_0, b_0), p_0) \)
    lie in \( \Op{fiber}_u(y) \), which is necessarily contractible,
    hence \( (a_0 , b_0) = (a, b) \) as desired.

    In the other direction, the assumption provides the missing properties to define a tentative inverse to \( \SigmaIsolate \);
    that this is the correct inverse follows as being isolated is a proposition (cf.~\autoref{is-prop-is-isolated}).
  \end{proof}
\end{lemma}

Sums of discrete types over a discrete base are themselves discrete,
and in this case \( \SigmaIsolate \) is trivially invertible:
\begin{corollary}\label{discrete-is-equiv-sigma-isolated}
  If \( A : \Type \) is discrete, and \( B : A \to \Type \) is a family of discrete types,
  then \( \SigmaIsolate_{A,B} \) is an equivalence.
  \begin{proof}
    Condition \ref{is-equiv-sigma-isolate-iff-isolated-pair-pair} of \autoref{is-equiv-sigma-isolate-iff-isolated-pair} is vacuously satisfied for discrete types.
  \end{proof}
\end{corollary}

Less trivially, we can state this locally for an isolated point in the base,
and obtain a generalization of \autoref{sum-embeddings-respect-isolated} from binary-
to arbitrary sums:
\begin{proposition}
  Let \( A : \Type \), \( B : A \to \Type \), and \( a : \Isolated{A} \).
  Then any \( b : B(a) \) is isolated if and only if \( (a, b) : \sum_{A} B \) is isolated.
  \begin{proof}
    Only the backward direction is non-trivial,
    and follows from \autoref{embedding-reflect-isolated}:
    since \( a \) is isolated,
    \( \lambda b.\, (a, b) \) is an embedding
    ---
    hence it reflects isolated points.
  \end{proof}
\end{proposition}

In general, however, it seems difficult to describe exactly when isolated points distribute this way.
In extreme cases, both \( A \) and \( B \) can be complicated types, whereas their sum \( \sum_A B \) is entirely trivial.
This applies in particular to the type of singletons, \( \Op{singl}(a_0) \DefEq \sum_{a : A} a_0 = a \):
\begin{proposition}\label{discrete-iff-is-equiv-singl-isolate}
  For all types \( A \), following are equivalent:
  \begin{enumerate}
    \item \( A \) is discrete.
    \item For all \( a_0 : A \), the map
      \(
        \SigmaIsolate_{A, a_0 = {\Blank}} : \sum_{a : \Isolated{A}} \Isolated{(a_0 = a)} \to \Isolated{\Op{singl}(a_0)}
      \)
      is an equivalence.
  \end{enumerate}
  \begin{proof}
    If \( A \) is discrete, then so are its path types, hence \( \SigmaIsolate_{A, a_0 = \Blank} \) is an equivalence by \autoref{discrete-is-equiv-sigma-isolated}.
    In the other direction, we can show that every \( a_0 : A \) is isolated:
    Since \( \Op{singl}(a_0) \) is a contractible type, its center \( (a_0, \Op{refl}) \) is isolated.
    Thus, by \autoref{is-equiv-sigma-isolate-iff-isolated-pair}, the first component \( a_0 \) must be isolated.
  \end{proof}
\end{proposition}
This lets us describe whether a type is discrete purely in terms of \( \SigmaIsolate \),
which will be useful in situations in which we have control over the types indexing \( \SigmaIsolate \).

\subsection{Removing points}


For any type \( A \) and point \( a_0 : A \) we define the subtype of \enquote{\( A \) with \( a_0 \) removed}
to be \( A \setminus a_0 \DefEq \sum_{a : A} a_0 \neq a \).
Adding a point to a type \( A \) and then removing it again yields an equivalence, as expected:
\begin{problem}\label{maybe-minus-nothing-equiv}
  Define an equivalence \( (A + 1) \setminus \Nothing \simeq A \).
\end{problem}

This is an instance of the more general case where removing a point from a sum
is the same as removing it from either side, and then taking the sum:
\begin{problem}\label{sum-remove-equiv}
  Given \( A, B : \Type \), define equivalences
  \begin{align*}
    (A + B) \setminus \Inl(a_0) &\simeq (A \setminus a_0) + B \\
    (A + B) \setminus \Inr(b_0) &\simeq A + (B \setminus b_0)
  \end{align*}
  for all points \( a_0 : A \) and \( b_0 : B \), respectively.
  \begin{construction}
    Define the obvious maps in either direction by case-analysis.
    These preserve inequalities since \( \Inl \) and \( \Inr \) are embeddings,
    and are inverses of each other as inequalities are always propositions.
  \end{construction}
\end{problem}

\begin{construction}[note={for \autoref{maybe-minus-nothing-equiv}}]
  The type \( (1 \setminus \bullet) \) is empty,
  so by \autoref{sum-remove-equiv}, \( (A + 1) \setminus \Nothing \simeq A + (1 \setminus \bullet) \simeq A \).
\end{construction}

Let us now consider the converse problem:
\emph{first} removing a point \( a_0 : A \), \emph{then} adding it back into the type.
Does this yield a type equivalent to \( A \)?
There is an obvious map \( \Replace_{a_0} : (A \setminus a_0) + 1 \to A \);
defined by cases
as
\(
  \Replace_{a_0}(\Just{(a, \mathunderscore)}) \DefEq a
\)
and
\(
  \Replace_{a_0}(\Nothing) \DefEq a_0
\).
Classically, this map is an isomorphism of sets;
the inverse simply maps \( a_0 \) to \( \bullet \).
In a univalent setting however, this replaces the higher path spaces of \( a_0 \)
with the trivial ones of \( \bullet \).
Consider for example the circle \( S^1 \) with \( \mathsf{base} : S^1 \).
Then \( S^1 \setminus \mathsf{base} \simeq 0 \),
hence \( (S^1 \setminus \mathsf{base}) + 1 \simeq 1 \not\simeq S^1 \):
removing \( \mathsf{base} \) removes the entire circle, which is \emph{not} contractible.
% This intui

We observe that we can replicate the classical behavior exactly when \( a_0 \) is an isolated point:

\begin{proposition}\label{isolated-minus-plus-equiv}
  Let \( a_0 : A \).
  The map \( \Replace_{a_0} \) is an equivalence typed \( (A \setminus a_0) + 1 \simeq A \)
  if and only if \( a_0 \) is isolated in \( A \).
  \begin{proof}
    First, assume that \( a_0 \) is isolated.
    We give a two-sided inverse \( g : A \to (A \setminus a_0) + 1 \) as follows.
    Define \( \Op{r} : \prod_{a : A} \Dec{(a_0 = a)} \to (A \setminus a_0) + 1 \)
    by
    \[
      \Op{r}(a, d) \DefEq
      \begin{cases}
        \Nothing & \text{if}\ d \JudgeEq \Op{yes}(\mathunderscore : a_0 = a) \\
        \Just{(a, h)} & \text{if}\ d \JudgeEq \Op{no}(h : a_0 \neq a)
      \end{cases}
    \]
    For \( i_0 : \Op{isIsolated}(a_0) \), let \( g(a) \DefEq \Op{r}(a, i_0(a)) \).
    By \autoref{is-prop-isolated-dec-path}, \( \Dec(a_0 = a) \) (the type of \( i_0(a) \)) is a proposition for all \( a : A \);
    we use this to ensure that \( g \) computes correctly:
    \begin{align*}
      g(a_0) &\JudgeEq \Op{r}(a_0, \Highlight{i_0(a_0)}) = \Op{r}(a_0, \Highlight{\Op{yes}(\Op{refl})}) \JudgeEq \Nothing \\
      g(a)   &\JudgeEq \Op{r}(a, \Highlight{i_0(a)}) = \Op{r}(a, \Highlight{\Op{no}(h))} \JudgeEq \Just{(a, h)}
        \quad \text{where}~h : a_0 \neq a
    \end{align*}
    Hence \( {\Replace_{a_0}} \circ g = \Op{id} \) and \( g \circ {\Replace_{a_0}} = \Op{id} \).

    For the converse, assume that \( \Replace_{a_0} \) is an equivalence.
    By \autoref{is-isolated-nothing}, \( \Nothing \) is isolated in \( (A \setminus a_0 ) + 1 \),
    so \autoref{is-isolated-respect-equiv} tells us that \( \Replace_{a_0}(\Nothing) \JudgeEq a_0 \) is isolated as well.
  \end{proof}
\end{proposition}

Later on, it will become necessary to understand how to remove points from \( \Sigma \)-types.
If we think of an \( A \)-indexed sum \( \sum_{a : A} B(a) \) as a generalization of binary sums \( B_0 + B_1 \),
then we expect removal to behave similarly:
Removing some \( (a, b) \) should remove \( b : B(a) \) from the \( a \)\textsuperscript{th}
summand, leaving all other summands unchanged.
This is indeed the case, as long as we put some restrictions%
\footnote{In~\cite[\href{https://www.cs.bham.ac.uk/~mhe/TypeTopology/UF.Sets.html\#is-h-isolated}{UF.Sets}]{EscardocontributorsTypeTopology}, \( a_0 \) is called \emph{\( h \)-isolated} if \( a_0 = a_0 \) is a proposition.}
on the paths of the indexing type \( A \):
\begin{problem}\label{sigma-remove}
  Let \( A : \Type \) and \( B : A \to \Type \)
  with points \( a_0 : A \) and \( b_0 : B(a_0) \),
  and assume \( p : \Op{isProp}(a_0 = a_0) \).
  There is a map
  \[
    \operatorname{\Op{\Sigma-remove}}_p :
    \big(\smashoperator{\sum_{a : A \setminus a_0}} B(a)\big) + \big(B(a_0) \setminus b_0\big)
      \to
    \big(\sum_{a : A} B(a) \big) \setminus (a_0 , b_0)
  \]
  \begin{construction}
    We define \( \operatorname{\Op{\Sigma-remove}}_p(x) \) by cases.
    Let
    \[
      \SigmaRemove_p(\Inl{(a, h_a, b)}) \DefEq ((a, b) , h^\prime_a),
    \]
    where \( h^\prime_a : (a_0 , b_0) = (a, b) \xrightarrow{\Op{cong}_{\Op{fst}}} a_0 = a \xrightarrow{h_a} \bot \).
    In the other case, let
    \[
      \SigmaRemove_p(\Inr{(b, h_b)}) \DefEq ((a_0, b) , h^\prime_b),
    \]
    and show \( h^\prime_b : (a_0 , b_0) \neq (a_0 , b) \) as follows:
    Assume to the contrary that \( p_b : (a_0 , b_0) = (a_0 , b) \).
    From this we obtain (dependent) paths \( p_b^1 : a_0 = a_0 \) and \( p_b^2 : b_0 =_{p_b^1} b \).
    Since \( a_0 = a_0 \) is a proposition, we know that \( p_b^1 = \Op{refl} \),
    hence \( b_0 = b \).
    This is contradictory since we are given \( h_b : b_0 \neq b \).
  \end{construction}
\end{problem}

This map is an equivalence whenever we can decide if we are removing from a chosen index \( a_0 \):
\begin{proposition}\label{is-equiv-sigma-remove}
  Let \( A : \Type \), \( B : A \to \Type \) with \( a_0 : A \) and \( b_0 : B(a_0) \).
  If \( a_0 \) is an isolated point of \( A \), then \( \SigmaRemove \) of \autoref{sigma-remove}
  is an equivalence.
  \begin{proof}
    First note that \( a_0 = a_0 \) is a proposition by \autoref{is-prop-isolated-path},
    thus the map is well-defined.
    We construct an inverse
    \[
      \SigmaRemove^{\Inv}
        :
      \Big(\sum_{a : A} B(a) \Big) \setminus (a_0 , b_0)
        \to
      \Big(\smashoperator{\sum_{a : A \setminus a_0}} B(a)\Big) + \big(B(a_0) \setminus b_0\big)
    \]
    as follows:
    Introduce \( a : A \), \( b : B(a) \) and \( h : (a_0 , b_0) \neq (a , b) \),
    then decide whether \( a_0 = a \) or not.
    If \( p : a_0 = a \), we map to \( \Inr{(\Op{subst}_B(p, a), h^\prime)} \),
    where \( h^\prime : b_0 \neq \Op{subst}_B(p, a) \),
    which we conclude from \( h \) and \( p \).
    In case that \( h : a_0 \neq a \) we map to \( \Inl{((a, h), b)} \) directly.
    It is straightforward to verify that these maps are inverses of each other.
  \end{proof}
\end{proposition}

\begin{tikzpicture}
  % \filldraw[
  %   rounded corners=10pt,
  %   fill=blue!30,
  %   draw=blue!80!black,
  %   thick,
  % ]
  %   (0,0) rectangle (7,4);

  \coordinate (O) at (-0.5,-0.5);

  \draw[->] (O) -- ++(0,1) node[above] {\( b : B(a) \)};
  \draw[->] (O) -- ++(1,0) node[right] {\( a : A \)};

  \foreach \x/\X in {0/6, 1/3, 2/5, 3/1, 4/4, 5/7} {
    \foreach \y in {1,...,\X} {
      \fill[black] (0.5+\x, \y*0.5) circle (2pt);
    }
  }
\end{tikzpicture}

\subsection{Grafting}

In order to derive the chain rule for derivatives,
\citeauthor{AbbottEtAl2005DataDifferentiating}~\cite{AbbottEtAl2005DataDifferentiating} investigate functions \( f : A \setminus a \to B \) that are defined on all but one inputs.
They call the process of extending \( f \) to all of \( A \) \emph{grafting}.
We have seen that in the presence of higher types, removal is only well-behaved for isolated points.
In order to obtain a good notion of grafting, we have to adjust the definitions accordingly.
In particular, we derive an induction principle for types
\( A \) with a chosen isolated point \( a_0 : \Isolated{A} \):
Functions out of \( A \) are exactly those out of \( A \setminus a_0 \), plus a chosen point \( b_0 : B \).
First, we define \emph{grafting}:
\begin{problem}
  For types \( A \) and \( B \), construct a function
  \[
    \Graft : \prod_{a_0 : \Isolated{A}}
      \big((A \setminus a_0 \to B) \times B \big)
        \to
      (A \to B)
  \]
\end{problem}
\begin{construction}
  Let \( a_0 : \Isolated{A} \), \( f : A \setminus a_0 \to B \) and \( b_0 : B \).
  Decide equality with \( a_0 \) to define \( \Graft_{a_0}(f, b_0) : A \to B \) as follows:
  \[
    \Graft_{a_0}(f, b_0) \DefEq
    \lambda a.\,
    \begin{cases}
      f(a, h) & \text{if}\ (h : a_0 \neq a) \\
      b_0 & \text{otherwise}
    \end{cases}
    \qedhere
  \]
\end{construction}

We adopt the notation
\( \GraftSyntaxX{f}{b_0}{a_0} \DefEq \Graft_{a_0}(f, b_0) \)
of \citeauthor{AbbottEtAl2005DataDifferentiating},
or simply \( \GraftSyntax{f}{b_0} \) if \( a_0 : \Isolated{A} \) is understood from context.

\begin{proposition}[note={\( \Graft \)-induction}]\label{graft-equiv}
  For \( A : \Type \) with \( a_0 : \Isolated{A} \),
  grafting has the following properties:
  \begin{enumerate}
    \item (Computation rules). For all \( f : A \setminus a_0 \to B \) and \( b_0 : B \):
      \begin{align*}
        \GraftSyntaxX{f}{b_0}{a_0}(a_0) &= b_0
          &
        &\text{and}
          &
        \adjustlimits \prod_{a : A} \prod_{h : a_0 \neq a} \GraftSyntaxX{f}{b_0}{a_0}(a) &= f(a, h)
      \end{align*}
    \item
      \(
        \Graft_{a_0} :
        \big((A \setminus a_0 \to B) \times B \big)
          \simeq
        (A \to B)
      \)
      is an equivalence of types.
  \end{enumerate}
  \begin{proof}
    The computation rules are straightforward, but crucially make use of \autoref{is-prop-isolated-dec-path}:
    the type \( \Dec(a_0 = a) \) is a proposition for all \( a \), and \( \Graft \) is defined by induction on this type.
    To show that \( \Graft \) is an equivalence, consider that any \( f : A \to B \) can be split
    into \( f \circ \Op{fst} : A \setminus a_0 \to B \) and \( f(a_0) : B \);
    the computation rules ensure that this is an inverse to \( \Graft \).
  \end{proof}
\end{proposition}

As presented here, \( \Graft \) characterizes non-dependent functions out of types with an isolated point.
This generalizes to a dependent induction principle, i.e.\@ an equivalence
\(
  \big( (\prod_{a : A \setminus a_0} B(a)) \times B(a_0) \big) \simeq \big(\prod_{a : A} B(a) \big)
\)
for families \( B \) over \( A \).
Although not necessary for later results in this paper,
we include the construction of the dependent induction principle in our formalization of the results.


\section{Derivatives of Containers}\label{derivatives}

Let us recall the notion of a container in type theory:
\begin{definition}
  A \emph{container} \( (\MkCont{S}{P}) \) consists of \emph{shapes} \( S : \Type \) and
  a family \( P : S \to \Type \) of \emph{positions}.
  We access shapes and positions via postfix projections
  \( (\MkCont{S}{P})_{\Sh} \DefEq S \) and \( (\MkCont{S}{P})_{\Ps} \DefEq P \).
\end{definition}

Containers were introduced to model polymorphic data types,
hence they are closed under products \( \times \), sums \( + \) and substitution \( \Subst{\Blank}{\Blank} \);
see \autoref{container-operations} for their definitions.
The constant container at a type \( S \) is \( \CConst{S} \DefEq (\MkCont{S}{0}) \);
products and sums form a monoidal structure with units \( \CConst{1} \) and \( \CConst{0} \).
The identity container \( \Id \DefEq (\MkCont{1}{1}) \)
is a unit for substitution, which is a non-symmetric monoidal product.

\begin{figure}
  \begin{align*}
    % Products
    (F \CTimes G)_{\Sh} &\DefEq S \times T
      &
    (F \CTimes G)_{\Ps} &\DefEq \lambda (s, t).\, P_s + Q_t
      \\
    % Sums
    (F \CPlus G)_{\Sh} &\DefEq S + T
      &
    (F \CPlus G)_{\Ps} &\DefEq
      \lambda
      \begin{cases}
        \Inl(s).\, P_s \\
        \Inr(t).\, Q_t
      \end{cases}
      \\
    % Substitution
    (\Subst{F}{G})_{\Sh} &\DefEq \sum_{s : S} (P_s \to T)
      &
    (\Subst{F}{G})_{\Ps} &\DefEq \lambda (s, f).\, \sum_{p : P_s} Q_{fp}
  \end{align*}
  \caption{%
    Operations on containers \( F \JudgeEq (\MkCont{S}{P}) \) and \( G \JudgeEq (\MkCont{T}{Q}) \).
  }%
  \label{container-operations}
\end{figure}

\begin{definition}
  Let \( F \JudgeEq (\MkCont{S}{P}) \) and \( G \JudgeEq (\MkCont{T}{Q}) \).
  The type of \emph{cartesian morphisms} between \( F \) and \( G \) is
  \[
    \Cart{F}{G} \DefEq \sum_{f : S \to T} \prod_{s : S} Q_{fs} \simeq P_s
  \]
  We denote the shape- and position components of a morphism \( {f : \Cart{F}{G}} \)
  by \( {f_{\Sh} : F_{\Sh} \to G_{\Sh}} \) and \( f_{\Ps} : \prod_{s : F_{\Sh}} G_{\Ps}(f_{\Sh}(s)) \simeq F_{\Ps}(s) \), respectively.
\end{definition}

In the remainder of this paper we will only consider cartesian morphisms.
We are going to drop \enquote{cartesian} in writing, but retain the notation \( \Cart{F}{G} \).
Morphisms of containers compose as expected, and together with an identity morphism \( \Op{id}_F : \Cart{F}{F} \)
they form a wild category \( \ContCart \):
composition is associative and unital, but we make no assumption on the truncation level of the hom-types \( \Cart{F}{G} \).
Note that this wild category is \emph{not} univalent in the na{\"i}ve sense:
The canonical map taking paths of containers \( F = G \) to categorical isomorphisms
(i.e. pairs \( f : \Cart{F}{G}, g : \Cart{G}{F} \) with chosen paths \( fg = \Op{id}_F \) and \( gf = \Op{id}_G \)),
is \emph{not} an equivalence, unless shapes and positions of the involved containers are sets.
Instead, we are going to use the following definition when comparing containers:

\begin{definition}
  A cartesian morphism \( (f, u) : \Cart{F}{G} \) is an \emph{equivalence} of containers
  if \( f : F_{\Sh} \to G_{\Sh} \) is an equivalence of types.
  We write \( F \CartEquiv G \) for the type of equivalences of containers.
\end{definition}
By an application of univalence, the type of equivalences \( F \CartEquiv G \) is equivalent to the type of paths, \( F = G \).
While we cannot prove it internally, we think of the wild category of containers as an \( (\infty,1) \)-category
in which \( F \CartEquiv G \) is the \enquote{correct} notion of weak equivalence,
representing the \( \infty \)-groupoid of paths \( F = G \).

In cases where we do care about the truncation level of shapes and positions,
we define the following subtypes of containers:
\begin{definition}
  A container \( (\MkCont{S}{P}) \) is \emph{\( (n,k) \)-truncated} if \( S \) is \( n \)-truncated,
  and \( P_s \) is \( k \)-truncated for all \( s : S \).
  Write \( \ContCart_{n,k} \) for the wild subcategory of \( (n,k) \)-truncated containers.
  A container is \emph{discrete} if \( P_s \) is a discrete type for all \( s : S \).
\end{definition}
Traditional set-based containers are \( (0,0) \)-truncated.
In particular, \( \ContCart_{0,0} \) forms a univalent 1-category in which a morphism being an isomorphism is a proposition equivalent to it being an equivalence of containers.
An example of containers of higher truncation level are \citeauthor{Gylterud2011}'s \emph{symmetric containers}~\cite{Gylterud2011}:
these have groupoids for shapes, and sets for positions, hence are exactly the \( (1,0) \)-truncated containers.

When constructing a morphism \( f : \Cart{F}{G} \), we will oftentimes factor it
through (equivalent) auxiliary containers
\begin{equation*}
  \begin{tikzcd}
    F \ar[r, -multimap, "f"] & G \\
    {F'} \ar[r, -multimap, "f'"{swap}] & {G'}
    \ar[from=1-1,to=2-1, -multimap, "\sim"{swap}]
    \ar[from=2-2,to=1-2, -multimap, "\sim"{swap}]
  \end{tikzcd}
\end{equation*}
This lets us separate the bureaucracy of bringing \( F \) and \( G \) into a comparable shape
from the act of defining an interesting morphism \( f' : \Cart{F'}{G'} \).
As a consequence, \( f \) is an equivalence of containers if and only if \( f' \) is,
which is often easier to characterize.

\subsection{Derivatives}

The derivative of a container \( G \) represents a type of
\( G \)-shaped trees in which a chosen subtree has been removed.
For traditional containers, this can be implemented as an operation \( G \mapsto \Der{G} \) by removing a chosen position over each shape:
Given \( G \JudgeEq (\MkCont{T}{Q}) \), \citeauthor{AbbottEtAl2005DataDifferentiating} define \( \Der{G} \)
to have as shapes pairs \( (t , q) : \sum_{T} Q \), over which the positions are \( Q_t \setminus q \).
To ensure that \( \Der \) is well-behaved, it is characterized by a universal property:
on a suitable subcategory of containers,\footnote{namely that of discrete containers} \( \Der \) extends to an endofunctor,
and one is interested in finding an adjunction \( \Maybe{\Blank} \dashv \Der \)
that describes morphisms into \( \Der{G} \):
If such an adjunction exists, then morphisms \( \Cart{F}{\Der{G}} \) are in 1-to-1 correspondence with morphisms of shape \( (f, u) : \Cart{\Maybe{F}}{G} \).
On positions, such morphisms are equivalences \( u_s : G_{\Ps}(fs) \simeq F_{\Ps}(s) + 1 \),
i.e.\@ maps that \enquote{avoid} the removed position \( u_s^{\Inv}(\Inr(\bullet)) : G_{\Ps}(fs) \).

As we have seen in the previous section, removing points from a type is a subtle process in a univalent setting:
a position \( q : G_{\Ps}(t) \) is not simply a discrete point, but comes with a potentially complicated type of paths around it.
If we wanted to encode morphisms into \( \Der{G} \) by the same universal property,
we would have to avoid the entire connected component around \( q \),
that is, find some type of \enquote{hole} \( H(q) \) such that \( G_{\Ps}(fs) \simeq F_{\Ps}(s) + H(q) \).
But this type now depends on \( q \) and its higher path structure,
and can no longer be expressed uniformly as a simple product with the fixed container \( \Id \).

From this, we can devise two ways forward:
Either we characterize the derivative in terms of a more fine-grained universal property
that takes the dependency on higher paths into account,
or we only take derivatives with respect to positions whose path types are more uniform.
For the purpose of this paper we take the second approach,
and define a derivative in terms of \emph{isolated} positions:
\begin{definition}
  The derivative of a container \( \Der{(\MkCont{S}{P})} \DefEq (\MkCont{S'}{P'}) \)
  has shapes \( S' \DefEq \sum_{s : S} \Isolated{P_s} \) and positions \( P'(s , p) \DefEq P_s \setminus p \).
\end{definition}

We can not only take the derivative of a container, but also act functorially on morphisms:
\begin{problem}
  Define a wild endofunctor \( \Der : \ContCart \to \ContCart \).
  That is,
  for all \( f : \Cart{F}{G} \), a morphisms \( \Der{f} : \Cart{\Der{F}}{\Der{G}} \),
  such that \( \Der(\Id_F) = \Id_{\Der{F}} \) and \(\Der(fg) = (\Der{f})(\Der{g}) \).
  \begin{construction}
    For any \( (f, u) : \Cart{(\MkCont{S}{P})}{(\MkCont{T}{Q})} \),
    there is a canonical morphism \( \Der{(f, u)} \DefEq (f' , u') : \Cart{\Der{(\MkCont{S}{P})}}{\Der{(\MkCont{T}{Q})}} \)
    obtained as follows:
    On shapes, the map \( f' : \sum_{s : S} \Isolated{P_s} \to \sum_{t : T} \Isolated{G_t} \)
    applies \( f \) to the first component and \( \Isolated{(u_s^\Inv)} : \Isolated{P_s} \simeq \Isolated{G_{fs}} \) to the second.
    On positions, \( u'_{s,p} : G_{fs} \setminus u_s^\Inv(p) \simeq F_s \setminus p \) is obtained from \( u_s \), which respects the removed point \( p \).
    (cf.\@ \autoref{is-isolated-respect-equiv}).
  \end{construction}
\end{problem}

Since isolated points always form a set, taking the derivative of a container preserves its truncation level
as long as shapes are at least sets, and positions are at least propositions:
\begin{proposition}
  For \( n \geq 0 \) and \( k \geq -1 \), the derivative of an \( (n, k) \)-truncated container is \( (n, k) \)-truncated.
  \begin{proof}
    Let \( (\MkCont{S}{P}) \) an \( (n, k) \)-truncated container.
    By \autoref{is-set-isolated} \( \Isolated{P_s} \) is a 0-truncated type and \( S \) is \( n \)-truncated,
    thus \( \Der{(\MkCont{S}{P})}_{\Sh} \JudgeEq \sum_{s : S} \Isolated{P_s} \) is \( n \)-truncated.
    Positions are \( k \)-types since \( P_s \setminus p \) embeds into \( P_s \).
  \end{proof}
\end{proposition}

Importantly, this turns \( \Der \) into an endofunctor on the 1-category of set-truncated containers,
without having to assume that the containers are discrete:
\begin{corollary}
  Taking derivatives is an endofunctor \( \Der : \ContCart_{0,0} \to \ContCart_{0,0} \).
\end{corollary}
We believe that analouges of this hold for higher truncation levels.
Symmetric containers, for example, form a bicategory \( \ContCart_{1,0} \),
and it should be straightforward (albeit tedious) to show that \( \Der \) restricts to a pseudo\-functor on this bicategory.

\subsection{Linear adjunction}

We now show that this generalized derivative operation is right-adjoint to \( \Maybe{\Blank} \),
verifying that it has the desired universal property.
We define this adjunction in terms of unit- and counit natural transformations,
and discuss how this relates to the original construction of \citeauthor{AbbottEtAl2005DataDifferentiating}.

\begin{problem}\label{derivative-adjunction}
  Define a wild adjunction \( (\eta, \varepsilon) : \Maybe{\Blank} \dashv \Der \),
  that is:
  \begin{enumerate}
    \item
      Two families of morphisms
      \[
        \eta : \prod_{F : \Cont} \Cart{F}{\Der{(F \CTimes \Id)}}
        \quad\text{and}\quad
        \varepsilon : \prod_{G : \Cont} \Cart{\Der{G} \CTimes \Id}{G}
      \]
    \item with fillers of naturality squares
      \begin{align*}
        &
          \begin{tikzcd}[ampersand replacement=\&]
            F
              \ar[r, -multimap, "\eta_F"]
              \ar[d, -multimap, "f"{swap}]
              \&
            \Der(\Maybe{F})
              \ar[d, -multimap, "\Der(\Maybe{f})"]
              \\
            G
              \ar[r, -multimap, "\eta_G"{swap}]
              \&
            \Der(\Maybe{G})
          \end{tikzcd}
        &
          &\text{and}
        &
        &
          \begin{tikzcd}[ampersand replacement=\&]
            \Maybe{\Der{F}}
              \ar[r, -multimap, "\varepsilon_F"]
              \ar[d, -multimap, "\Maybe{\Der{f}}"{swap}]
              \&
            F
              \ar[d, -multimap, "f"]
              \\
            \Maybe{\Der{G}}
              \ar[r, -multimap, "\varepsilon_G"{swap}]
              \&
            G
          \end{tikzcd}%
      \end{align*}
      for all \( f : \Cart{F}{G} \),
    \item and zigzag-diagrams
      \begin{align*}
        &
        \begin{tikzcd}[ampersand replacement=\&, column sep=tiny, row sep=large]
          \Maybe{F} \& \& \Maybe{F} \\
                    \& \Maybe{(\Der{(\Maybe{F})})} \& %
          \ar[from=1-1, to=1-3, -multimap, "\Op{id}"]
          \ar[from=1-1, to=2-2, -multimap, "\Maybe{{\eta_F}}"{'}]
          \ar[from=2-2, to=1-3, -multimap, "\varepsilon_{\Maybe{F}}"{'}]
        \end{tikzcd}
        &
          &\text{and}
        &
        &
        \begin{tikzcd}[ampersand replacement=\&, column sep=tiny, row sep=large]
                    \& \Der{(\Maybe{\Der{G}})} \& \\
          \Der{G} \& \& \Der{G} %
          \ar[from=2-1, to=2-3, -multimap, "\Op{id}"]
          \ar[from=2-1, to=1-2, -multimap, "\eta_{\Der{G}}"]
          \ar[from=1-2, to=2-3, -multimap, "\Der{(\varepsilon_{G})}"]
        \end{tikzcd}
      \end{align*}
  \end{enumerate}
  \begin{construction}
    Let \( F \JudgeEq (\MkCont{S}{P}) \) and define \( \eta_F : \Cart{F}{\Der{(\Maybe{F})}} \).
    On shapes, \( {\eta_F^{\Sh}} : S \to \sum_{(s, \Blank) : S \times 1} \Isolated{(P_s + 1)} \)
    sends \( s \) to \( (s, \bullet) \) and \( \mathsf{nothing} \);
    the latter is isolated by \autoref{is-isolated-nothing}.
    On positions, define \( \eta_F^{\Ps} : \prod_{s : S} (P_s + 1) \setminus \Nothing \simeq P_s \)
    as in \autoref{maybe-minus-nothing-equiv}.

    Let \( G \JudgeEq (\MkCont{T}{Q}) \); define the counit \( \varepsilon_G : \Cart{\Maybe{(\Der{G})}}{G} \) as follows:
    on shapes, \( \varepsilon_G^{\Sh} : \sum_{t : T} \Isolated{Q_t} \times 1 \to T \) is the first projection.
    On positions the equivalence
    \(
      \varepsilon_G^{\Ps}(t , q) : Q_t \simeq (Q_t \setminus q) + 1
    \)
    is given by \autoref{isolated-minus-plus-equiv} for all \( t : T \) and \( q : \Isolated{Q_t} \).

    To construct the zigzag-fillers,
    we apply the necessary extensionality principles for functions, equivalences and sum types.
    We are left to construct paths that are almost \( \Op{refl} \);
    only some proofs of isolation and removal need to be compared up to propositional equality.
    Construction of the naturality squares for \( \eta \) and \( \varepsilon \) is done similarly.
  \end{construction}
\end{problem}

In their original construction, \citeauthor{AbbottEtAl2005DataDifferentiating} only establish isomorphisms between hom-sets
\( \Cart{\Maybe{F}}{G} \) and \( \Cart{F}{\Der{G}} \), natural in \( F \).
This falls short of defining a proper adjunction since \( \Der{G} \) is left undefined for non-discrete containers \( G \).
We can however complete the search for a suitable subcategory of differentiable containers \cite[14]{AbbottEtAl2005DataDifferentiating}:
Our derivative is defined functorially for \emph{all} containers,
and restricting the above wild adjunction to set-truncated containers yields the following:
\begin{theorem}
  In the 1-category of set-truncated containers \( \ContCart_{0,0} \), \( \Der \) is right-adjoint to tensoring \( \Blank \CTimes \Id \). \qed
\end{theorem}
From this, we can extract the familiar natural isomorphism of hom-sets in \( \ContCart_{0,0} \).
In fact, the same argument lets us obtain a natural equivalence of hom-types for arbitrary containers,
which otherwise would be somewhat tedious to establish:
there is an equivalence \( (\Cart{F}{\Der{G}}) \simeq (\Cart{\Maybe{F}}{G}) \) natural in \( F, G : \ContCart \),
with underlying map\footnote{%
  Interestingly, the proof of \cite[{Theorem 5.1}]{AbbottEtAl2005DataDifferentiating} already uses \( {\Blank}^{\sharp} \) to show naturality of the hom-set isomorphism in \( F \)!
}
\begin{align*}
  {\Blank}^\sharp &: (\Cart{F}{\Der{G}}) \to (\Cart{\Maybe{F}}{G}) \\
  f^{\sharp} &\DefEq \varepsilon_G \circ (\Maybe{f})
\end{align*}
That \( {\Blank}^{\sharp} \) is an equivalence
means that for all \( { g : \Cart{\Maybe{F}}{G} } \)
the type of fibers
\(
  \sum_{f : \Cart{F}{\Der{G}}} \varepsilon_G \circ (\Maybe{f}) = g
\) is contractible.
This expresses the \emph{property} of \( \varepsilon_G \) being a universal arrow from \( \Maybe{\Blank} \) to \( G \):
any \( g \) factors uniquely through \( \varepsilon_G \).
Hence, the data of the wild adjunction in \autoref{derivative-adjunction} is uniquely determined, up to our choice of counit.

\subsection{Laws of Derivates}

Derivatives of containers earn their name by observing how they interact with other operations on containers:
derivatives of constants are zero, derivatives of sums and products follow the familiar sum- and product rules,
and the derivative of a composite container is characterized by a chain-rule.
Let us now investigate to which extent our derivative still respects this structure.

It is easy to see that derivatives of constants are always zero,
and that \( \Der{\Id} \) is the constant \( \CConst(1) \).
Both factor through the following observation:
\begin{proposition}\label{derivative-prop-trunc}
  Let \( S : \Type \) and \( P : S \to \Op{Prop} \).
  There is an equivalence of containers
  \[
    { \Der{(\MkCont{S}{P})} }
      \CartEquiv
    { (\MkCont{{\textstyle \sum_{S} P }}{0}) }
  \]
  In particular, we have
  \( \Der{(\Id)} \CartEquiv \CConst(1) \)
  and \( \Der(\CConst(A)) \CartEquiv \CConst(0) \) for all \( A : \Type \).
  \begin{proof}
    Since \( P_s \) is a proposition, we know that \( \Isolated{P_s} \simeq P_s \) and \( P_s \setminus p \simeq 0 \).
    Thus,
    \begin{align*}
      \Der{(\MkCont{S}{P})}
        &\CartEquiv (\MkCont{((s, p) : \textstyle\sum_{s : S} \Isolated{P_s})}{P_s \setminus p})
          \\
        &\CartEquiv (\MkCont{((s, p) : \textstyle\sum_{s : S} P_s)}{0})
          \qedhere
    \end{align*}
  \end{proof}
\end{proposition}

Similarly, we convince ourselves that \( \Der \) distributes over (binary) sums,
and that derivatives of products follow a Leibniz rule:
\begin{proposition}\label{sum-product-rule}\label{sum-rule}\label{leibniz-rule}
  For containers \( F, G \), the following hold:
  \begin{enumerate}
    \item Sum rule: \( {\Der{(F \CPlus G)}} \CartEquiv {\Der{F} \CPlus \Der{G}} \)
    \item Leibniz rule: \( {\Der{(F \CTimes G)}} \CartEquiv {(\Der{F} \CTimes G) \CPlus (F \CTimes \Der{G})} \)
  \end{enumerate}
  \begin{proof}
    Let \( F \JudgeEq (\MkCont{S}{P}) \) and \( G \JudgeEq (\MkCont{T}{Q}) \).
    Both equivalences are established like in the discrete setting
    (cf.~\cite[{Proposition 6.3 and 6.4}]{AbbottEtAl2005DataDifferentiating}),
    with one exception:
    to derive the Leibniz rule,
    one needs to show that isolated points distribute over binary sums in
    \[
      \sum_{s : S} \sum_{t : T} \Isolated{(P_s + Q_t)}
        \simeq
      \sum_{s : S} \sum_{t : T} \Isolated{P_s} + \Isolated{Q_t},
    \]
    which is done via \autoref{isolated-sum-equiv}.
  \end{proof}
\end{proposition}

To some extent we can also solve differential equations involving \( \Der \).
In particular, given a container \( F \), we can ask if it has an anti-derivative,
i.e. some \( G \) for which \( \CartIso{\Der{G}}{F} \).
Interestingly, \( \Der \) has fixed-points, that is containers that are their own derivative.
The prototypical example of such a fixed-point is the container of finite multisets, or \emph{bags}.\todo{Say that we're reproducing \cite[{Ex.~3.6.1}]{Gylterud2011}, but internally.}
Recall that a type is considered finite
if there is some \( n : \mathbb{N} \) for which it is merely equivalent to \( \Op{Fin}(n) \).%
% \footnote{This is the notion of Bishop-finiteness~\cite[{Definition~4.4}]{FruminEtAl2018FiniteSetsHomotopy}, one of many notions of finiteness in a constructive setting.}
\footnote{Specifically, such types are called Bishop-finite~\cite[{Definition~4.4}]{FruminEtAl2018FiniteSetsHomotopy}.}
The universe of finite sets,
\( \FinSet \DefEq \sum_{X : \Type} \sum_{n : \mathbb{N}} \lVert X \simeq \Op{Fin}(n) \rVert \),
comes with a map \( \El \DefEq \Op{fst} : \FinSet \to \Type \) projecting out the underlying type.
Together, these form the container of \emph{bags}, \( \Op{Bag} \DefEq (\MkCont{\FinSet}{\El}) \).
As written, the shapes of this container quantify over all types, hence live in a higher universe.
There are however equivalent small replacements of this type,
such as the one given by \citeauthor{FinsterEtAl2021CartesianBicategoryPolynomial} in \cite[Theorem~25]{FinsterEtAl2021CartesianBicategoryPolynomial}.

\begin{proposition}\label{bag-fixed-point}
  The bag-container is a fixed-point of derivation:
  there is an equivalence \( \CartIso{\Der{\Op{Bag}}}{\Op{Bag}} \).
  \begin{proof}
    First, note that finite sets are closed under addition and removal of points:
    if \( X \) is finite, then so are \( X + 1 \) and \( X \setminus x \) for all \( x : X \).
    On shapes, we construct an equivalence
    \(
      \sum_{X : \FinSet} \Isolated{\El(X)} \simeq \FinSet
    \)
    from mutually inverse functions \( f \) and \( g \).
    From left to right, define \( f(X, x_0) \DefEq X \setminus x_0 \);
    the other way let \( g(X) \DefEq (X + 1 , \Nothing) \).
    By univalence, finite sets are equal if their carrier types are equivalent,
    so \( f \) and \( g \) are inverses by \autoref{isolated-minus-plus-equiv} and \autoref{maybe-minus-nothing-equiv}.
    Given \( X : \FinSet \) and \( x_0 : \El(X) \), positions are related by the identity equivalence,
    that is
    \(
      \Op{Bag}_{\Ps}(f(X, x_0)) = \El(f(X, x_0)) = X \setminus x_0 = \Der{\Op{Bag}}_{\Ps}(X, x_0)
    \).
  \end{proof}
\end{proposition}

Unlike in classical analysis, where the exponential function is the unique solution to
the differential equation \( f^{\prime} = f \) with initial condition \( f(0) = 1 \),
the situation for containers is more nuanced:
While \( \Op{Bag} \) is a solution for \( \CartIso{\Der{F}}{F} \) such that \( { F[\CConst{0}] \CartEquiv \CConst{1} } \),
it is far from being the only one.
This is not entirely unexpected:
containers are closely related to Joyal's \emph{combinatorial species}
(as discussed e.g.~in Yorgey's thesis \cite[67]{Yorgey2014CombinatorialSpeciesLabelled}),
and these are known to have many non-isomorphic solutions even for simple differential equations,
as shown by Labelle in~\cite{Labelle1986combinatorialdifferentialequations}.

Modulo size issues, the proof of \autoref{bag-fixed-point} goes through for any subuniverse
of types closed under addition and removal of single points:
\begin{proposition}
  Let \( P : \Type \to \Op{Prop} \) a predicate
  for which for all \( A : \Type \), \( P(A) \) implies both \( P(A + 1) \)
  and \( \prod_{a : A} P(A \setminus a) \).
  This defines a container
  \( \Op{Bag}_P \DefEq (\MkCont{\sum_{\Type} P}{\Op{fst}}) \)
  such that \( \CartIso{\Der{\Op{Bag}_P}}{\Op{Bag}_P} \). \qedhere
\end{proposition}
For the details of the proof we refer the reader to the formalization;
there one can find an example of this applied to subcountable sets
defined by the predicate \( P(A) \DefEq \lVert A \hookrightarrow \mathbb{N} \rVert \).

\subsection{The Chain Rule}

We expect the derivative of a composite of containers to satisfy a version of the chain rule
\( (f \circ g)^{\prime} = (f^{\prime} \circ g) \cdot g^{\prime} \).
In our setting, substitution \( \Subst{\Blank}{\Blank} \) takes on the role of composition,
and for discrete containers, \citeauthor{AbbottEtAl2005DataDifferentiating} show that
\( \Der{\Subst{F}{G}} \) is indeed isomorphic to \( \Subst{(\Der{F})}{G} \CTimes \Der{G} \).
Attempting to lift their proof to untruncated containers,
we run into difficulties:
While it is possible to define a morphism from one to the other,
it is not immediately clear that an inverse exists.
Precisely, we obtain the following \emph{directed} or \emph{lax} chain rule:
\begin{problem}[note={The lax chain rule}]\label{lax-chain-rule}
  For any two containers \( F, G \), define a morphism
  \[
    \Op{chain}_{F,G} :
    \Cart%
      { \Subst{(\Der{F})}{G} \CTimes \Der{G} }%
      {\Der{(\Subst{F}{G})}}
  \]
\end{problem}
\begin{construction}[note={for \autoref{lax-chain-rule}}]
  Let \( F \JudgeEq (\MkCont{S}{P}) \) and \( G \JudgeEq (\MkCont{T}{Q}) \).
  As usual, we have to construct a map on shapes and an equivalence of positions.
  On shapes, our goal is a map
  \[
    \big(\sum\nolimits_{(s, p) : \sum_{s : S} \Isolated{P_s}} (P_s \setminus p \to T) \big)
      \times
    \sum_{t : T} \Isolated{Q_t}
      \to
    \Big(
      \sum\nolimits_{(s, f) : \sum_{s : S} (P_s \to T)} \Isolated{\big( \sum_{p : P_s} Q_{fp} \big)}
    \Big)
  \]
  Let us first reshape the left side by some equivalences.
  By re-associating the sums, we obtain
  \begin{align*}
      %
    &\mathrel{\hphantom{\simeq}}
      \big(\sum\nolimits_{(s, p) : \sum_{s : S} \Isolated{P_s}} P_s \setminus p \to T \big)
        \times
      \sum_{t : T} \Isolated{Q_t}
    \\
    &\simeq
      \sum\nolimits_{(s, p) : \sum_{s : S} \Isolated{P_s}} \sum\nolimits_{(f, t) : (P_s \setminus p \to T \times T)} \Isolated{Q_t}
    % &\simeq
    %   \adjustlimits\sum_{s : S} \sum_{p : \Isolated{P_s}} \smashoperator[r]{\sum_{(\Blank, t) : (P_s \setminus p \to T \times T)}} \Isolated{Q_t}
  \intertext{%
    The induction principle for \( \Graft \) tells us that the types \( (P_s \setminus p \to T) \times T \) and \( P_s \to T \) are equivalent (\autoref{graft-equiv}),
    thus we simplify to
  }
    &\simeq
      \sum\nolimits_{(s, p) : \sum_{s : S} \Isolated{P_s}} \sum\nolimits_{f : P_s \to T} \Isolated{Q_{fp}}
  \intertext{%
    By permuting the sum yet again, we are left with
  }
    &\simeq
      \sum\nolimits_{(s, f) : \sum_{s : S} (P_s \to T)} \big( \sum_{p : \Isolated{P_s}} \Isolated{(Q_{fp})} \big)
  \end{align*}
  Denote this equivalence by \( \lambda \).
  Now, the left and the right only differ in
  \begin{align*}
    \sum_{p : \Isolated{P_s}} \Isolated{(Q_{fp})}
      \qquad\text{vs.}\qquad
    \Isolated{\big( \sum_{p : P_s} Q_{fp} \big)}
  \end{align*}
  \Autoref{sigma-isolate} gives us a map
  \(
    \SigmaIsolate_{P_s,Q_{f(\Blank)}} :
      \sum_{p : \Isolated{P_s}} \Isolated{(Q_{fp})}
        \to
      \Isolated{\big( \sum_{p : P_s} Q_{fp} \big)}
  \),
  hence
  \(
    \cramped{
      \Op{chain}_{F,G}^{\Sh} \DefEq
        \Op{\Sigma}(\Op{id}, \SigmaIsolate_{P_s,Q_{f(\Blank)}}) \circ \lambda
    }
  \).

  To construct the equivalence on positions,
  let \( s : S \), \( p_0 : \Isolated{P_s} \), \( f : P_s \setminus p_0 \to T \), \( t : T \) and \( q_0 : \Isolated{Q_t} \).
  Our goal becomes to construct an equivalence
  \[
      \big(
        \sum_{p : P_s} Q_{\GraftSyntaxX{f}{t}{p_0}(p)}
      \big) \setminus (p_0 , q_0)
    \simeq
      \big(
        \smashoperator{\sum_{p : P_s \setminus p_0}} Q_{f(p)}
      \big)
        +
      (Q_t \setminus q_0),
  \]
  which we obtain from \autoref{is-equiv-sigma-remove},
  and by applying the computation rules of grafting to \( \GraftSyntaxX{f}{t}{p_0} : P_s \to T \).
\end{construction}

Note that the above proof essentially factors \( \Op{chain}_{F,G} \) into
\[
  \begin{tikzcd}[column sep=large]
    { \Subst{(\Der{F})}{G} \CTimes \Der{G} }%
      \ar[r, -multimap, "\Op{chain}_{F,G}"]
      \ar[d, -multimap, "\sim"{swap}]
      &
    {\Der{(\Subst{F}{G})}}
      \\
    H_0
      \ar[r, -multimap, "\eta"{swap}]
      &
    H_1
      \ar[u, -multimap, "\sim"{swap}]
  \end{tikzcd}
\]
in which
\(
  \eta_{\Sh} :
    { \sum_{s : S} \sum_{f : P_s \to T} \sum_{p : \Isolated{P_s}} \Isolated{Q_{fp}} }%
      \to
    { \sum_{s : S} \sum_{f : P_s \to T} \Isolated{\big( \sum_{p : P_s} Q_{fp} \big)} }
\)
applies \( \SigmaIsolate_{P_s,Q_{f(\Blank)}} \).
We will make this factorization explicit in the derivation of the chain rule for indexed containers (\autoref{binary-chain-rule}),
but for now we can record the following fact:
\begin{proposition}\label{strong-chain-rule-iff-is-equiv-sigma-isolate}
  Let \( F = (\MkCont{S}{P}) \) and \( G = (\MkCont{T}{Q}) \).
  The following are equivalent propositions:
  \begin{enumerate}
    \item \label{strong-chain-rule-iff-is-equiv-sigma-isolate-1}
      \( \Op{chain}_{F,G} \) is an equivalence of containers
    \item \label{strong-chain-rule-iff-is-equiv-sigma-isolate-2}
        \( \SigmaIsolate_{P_s,Q_{f(\Blank)}} \) is an equivalence for all \( s : S \) and \( f : P_s \to T \)
    \item \label{strong-chain-rule-iff-is-equiv-sigma-isolate-3}
      For all \( s : S \) and \( f : P_s \to T \), if \( (p, q) : \sum_{p : P_s} Q_{fp} \) is isolated,
      then both \( p \) and \( q \) are isolated.
  \end{enumerate}
  \begin{proof}
    Equivalence of the first two points follows from inspection of the definition of \( \Op{chain} \) in terms of \( \SigmaIsolate \).
    Equivalence with the last point is \autoref{is-equiv-sigma-isolate-iff-isolated-pair}.
  \end{proof}
\end{proposition}

We say that the chain rule for \( F \) and \( G \) is \emph{strong} if \( \Chain_{F,G} \) is an equivalence of containers;
the above lets us immediately see when we can and cannot expect this to be the case.
Firstly, we recover a strong chain rule for \emph{discrete} containers, as expected by
\cite[{Proposition~6.6}]{AbbottEtAl2005DataDifferentiating}:
\begin{theorem}\label{discrete-strong-chain-rule}
  For discrete containers \( F, G \), \( \Op{chain}_{F,G} \) is an equivalence.
  In particular, it is an isomorphism in the 1-category of set-truncated containers.
  \begin{proof}
    The positions of \( F \) and \( G \) are discrete,
    so by \autoref{discrete-is-equiv-sigma-isolated}
    \( \SigmaIsolate \) is an equivalence.
    By the previous proposition, is \( \Op{chain}_{F,G} \) is an equivalence of containers.
  \end{proof}
\end{theorem}

Secondly, we conclude that globally having a strong chain rule is an inherently classical property:
\begin{theorem}\label{globally-discrete-iff-strong-chain-rule}
  The following are equivalent propositions:
  \begin{enumerate}
    \item \emph{every} type is discrete
    \item for all containers \( F \) and \( G\), \( \Op{chain}_{F,G} \) is an equivalence
  \end{enumerate}
  \begin{proof}
    If every type is discrete, then so is every container, hence the chain rule is always an equivalence by \autoref{discrete-strong-chain-rule}.
    In the other direction,
    use \autoref{discrete-iff-is-equiv-singl-isolate} to show that any given type \( A \) is discrete
    ---
    that is, given some \( a_0 : A \), prove that \( \SigmaIsolate_{A, {a_0 = \Blank}} \) is an equivalence.
    We do so by applying \autoref{strong-chain-rule-iff-is-equiv-sigma-isolate} to containers
    \( F \DefEq (\MkCont{1}{A}) \) and \( G \DefEq (\MkCont{(a : A)}{a_0 = a}) \).
  \end{proof}
\end{theorem}

As a consequence, globally assuming that the chain rule is strong is inconsistent in the presence of types of higher truncation level:
The circle \( S^1 \) is provably not discrete (if it were, it would be a set!),
hence
\(
  \neg ( \prod_{F, G : \Cont} \Op{isEquiv}(\Op{chain}_{F,G}) )
\).

If instead we restrict ourselves to the world of sets,
then we can conclude that a globally strong chain rule exists if and only if arbitrary equalities are decidable:
\begin{corollary}
  The following are equivalent:
  \begin{enumerate}
    \item Every set is discrete.
    \item In the 1-category of set-truncated containers, \( \Chain_{F,G} \) is an isomorphism
      for all pairs of containers \( F \) and \( G \).
  \end{enumerate}
  \begin{proof}
    In the 1-category \( \ContCart_{0,0} \), being an isomorphism is a proposition,
    which is equivalent to being an equivalence of containers.
    The claim follows by inspection of the proof of \autoref{globally-discrete-iff-strong-chain-rule},
    and ensuring that the same argument applies even when all types involved are sets.
  \end{proof}
\end{corollary}

We have seen that our definition of derivative behaves nicely even in the presence of non-discrete types,
in that we retain the ways it interacts with sums and products (\autoref{sum-product-rule}).
Its interaction with substitution, however, is more subtle.
While we do obtain a chain rule, it is now \emph{directed} or \emph{lax} (\autoref{lax-chain-rule}),
and whether we can strengthen it to an equivalence depends on the pair of containers involved (\autoref{strong-chain-rule-iff-is-equiv-sigma-isolate}).
Indeed, assuming the latter for any pair of containers is inconsistent in the presence of higher inductive types such the circle \( S^1 \).

\subsection{Derivatives of Fixed Points}

Containers let us model inductive data types as fixed-points to substitution \( \Subst{\Blank}{\Blank} \):
if \( F \) describes the branching of an inductive data type, then there is a corresponding
container \( \mu F \) such that \( \CartIso{\Subst{F}{\mu F}}{\mu F} \).
This lets us turn informal recursive specifications such as \( \Op{List}(X) = 1 + X \times \Op{List}(X) \)
into a precise description of \( \Op{List} \) as a container.
\Citeauthor{AbbottEtAl2005DataDifferentiating} prove that the derivative of a fixed-point point container
is itself a fixed point: \( \Der{(\mu F)} \) is equivalent to \( \mu F' \) for some \( F' \) derived from \( F \).
To do so, they recognize that each \( \mu F \) essentially behaves like repeated substitution \( \Subst{F}{\Subst{F}{\Subst{F}{\ldots}}} \),
and infer a fixed-point rule from the chain rule.

Our goal is to derive a similar fixed-point rule for our generalized derivative,
ideally by deriving it from the chain-rule as well.
This begs the question: if in general the chain-rule rule is lax,
is it ever possible to have a strong fixed-point rule?\todo{Spoil the answer to this question here.}
It appears that the answer to this question is \enquote{it depends}:
we are able to show (cf.~\autoref{strong-mu-rule-iff-strong-chain-rule})
that the fixed-point rule for an \( I + 1 \)-indexed container \( F \) is strong if and only if the chain rule
between \( F \) and \( \mu F \) is.

\subsubsection{Indexed Containers}

To define and reason about an operation \( \mu \) taking a container to a fixed-point,
we first have to recall the definition indexed containers.
These encode data types polymorphic in more than one variable,
which are necessary to transcribe fixed-point description of types such as
\( \Op{List}(X) = \mu Y.\, 1 + X \times Y \)
into actual containers.
\begin{definition}[note={Indexed containers}]
  Let \( I : \Type \). The type of \emph{\( I \)-ary} or \emph{\( I \)-indexed containers} is
  \(
    \Cont_{I} \DefEq \sum_{S : \Type} (I \to S \to \Type)
  \).
\end{definition}
Each index \( i : I \) corresponds to a variable in the type that a container describes.
For example, a type like \( F(X, Y) = 1 + X \times Y \) would be encoded as a binary container, indexed by the type \( 2 \).
Ordinary containers correspond to unary containers, \( \Cont_{1} \).

Cartesian morphisms between containers \(F , G : \Cont_{I} \) are defined by ranging over all indices \( I \):
\[
  \Cart{F}{G} \DefEq \smashoperator[l]{\sum_{f : F_{\Sh} \to G_{\Sh}}} \prod_{i : I, s : F_{\Sh}} G_{\Ps}(i, f(s)) \simeq F_{\Ps}(i, s)
\]
Together, \( I \)-indexed containers again assemble into a wild category, \( \ContCart_{I} \).
For \( F : \Cont_I \), denote by \( \Wk{F} : \Cont_{I+1} \) the inclusion into containers with one more variable
given by
\[
  \Wk{F}_{\Ps}(\Just{i}) \DefEq F_{\Ps}(i), \quad \Wk{F}_{\Ps}(\Nothing) \DefEq 0
\]
To aid readability, we denote an \( I+1 \)-indexed container \( (\MkCont{S}{\bar{P}}) \) by \( (\MkCont{S}{\vec{P}, P}) \),
where \( \vec{P}_i \DefEq \bar{P}_{\Just(i)} \) and \( P \DefEq \bar{P}_{\Nothing} \).
In particular, \( \Wk{(\MkCont{S}{P})} = (\MkCont{S}{P,0}) \).
For \( i : I \), the \( i \)th projection container is \( \pi_i \DefEq (\MkCont{1}{\lambda j\,\Blank.\,{i = j}}) : \Cont_I \).
If \( i \) is isolated, then \( i = j \) is a decidable proposition for any \( j : I \);
in this case \( \pi_i \) is equivalent to a container whose \( i \)th type of positions is \( 1 \),
and \( 0 \) for any other direction.
As expected, indexed containers are closed under constants \( \CConst(A) \), sums \( (\CPlus) \) and products \( (\CTimes) \);
Substitution generalizes to an operation that places a container inside the positions at index \( \Nothing : I + 1 \) of a \( I + 1 \)-indexed container:
\begin{definition}
    Substitution is an operation \( \Subst{\Blank}{\Blank} : \Cont_{I+1} \to \Cont_{I} \to \Cont_{I} \),
    defined as follows:
    \begin{align*}
      \Subst{(\MkCont{S}{\vec{P}, P})}{(\MkCont{T}{Q})}_{\Sh} &\DefEq \sum_{s : S} P(s) \to T
        \\
      \Subst{(\MkCont{S}{\vec{P}, P})}{(\MkCont{T}{Q})}_{\Ps} &\DefEq \lambda (s, f).\, \vec{P}_i(s) + \smashoperator{\sum_{p : P(s)}} Q_i(fp)
        \qedhere
    \end{align*}
\end{definition}
For any fixed \( F : \Cont_{I+1} \), \( F[-] \) is a wild endofunctor of \( I \)-ary containers.

We can now lift derivatives to indexed containers,
and define a derivative for each index, as long as equality with that index is decidable:
\begin{definition}[note={Derivative of \( I \)-ary containers}]
  Let \( F \JudgeEq (\MkCont{S}{P}) : \Cont_I \) and \( i : \Isolated{I} \).
  The \( i \)th derivative \( {\Der_i}F \DefEq (\MkCont{S'}{P'}) : \Cont_I \) is defined as follows:
  \begin{align*}
    S' &\DefEq \sum_{s : S} \Isolated{P_i(s)} \\
    P'(j, s, p) &\DefEq
      \begin{cases}
        P_i(s) \setminus p & \text{if } i = j \\
        P_j(s) & \text{otherwise}
      \end{cases}
      \qedhere
  \end{align*}
\end{definition}

In order to reduce visual clutter, we investigate only fixed-points of \emph{binary} containers, indexed by \( 2 = 1 + 1 \).
However, our arguments apply to \( I + 1 \)-indexed containers in general.
In particular, we write \( \Der_0 \) and \( \Der_1 \) for the two possible derivatives of a binary container,
that is \( \Der_{\Inl(\bullet)} \) and \( \Der_{\Inr(\bullet)} \).
First, we obtain a lax chain rule for binary containers.
As promised in the discussion of \autoref{lax-chain-rule},
we factor the problem into smaller steps that make it obvious why, in general, this rule is not invertible:
\begin{problem}[note={Lax chain rule for binary containers}]\label{binary-chain-rule}
  Let \( F : \Cont_2 \) and \( G : \Cont_1 \).
  Define a cartesian morphism
  \[
    \Op{chain}_{F,G}
      :
    \Cart%
      {{\Der_0{F}}[G] \CPlus \big( {\Der_1{F}}[G] \CTimes \Der{G} \big)}%
      {\Der{(F[G])}}
  \]
  \begin{construction}
    \renewcommand*{\L}{\Op{L}}
    \newcommand*{\R}{\Op{R}}
    Let \( F \JudgeEq (\MkCont{S}{P}) \), \( G \JudgeEq (\MkCont{T}{Q}) \).
    We define auxiliary containers \( H_1, H_2 \) and factor the morphism into the following:
    \[
      \begin{tikzcd}
        \L \DefEq {{\Der_0{F}}[G] \CPlus \big( {\Der_1{F}}[G] \CTimes \Der{G} \big)}
          \ar[r, -multimap, "\sim"] &
        H_1
          \ar[r, -multimap, "\eta"] &
        H_2
          \ar[r, -multimap, "\sim"] &
        {\Der{(F[G])}}
        \eqcolon \R
      \end{tikzcd}
    \]
    Following the argument in \autoref{lax-chain-rule} and some type yoga,
    we see that type of shapes of the container to the left is equivalent to
    \begin{align}
      U_1 \DefEq
        \sum_{s : S} \sum_{f : P_1(s) \to T} \Isolated{P_0(s)} + \smashoperator{\sum_{p : \Isolated{P_1(s)}}} \Isolated{Q(fp)}
          \label{binary-chain-rule-shape-equiv}
    \end{align}
    Denote this equivalence by \( f_1 : {\L}_{\Sh} \simeq U_1 \), and define \( H_1 \DefEq (\MkCont{U_1}{{\L}_{\Ps} \circ f_1^{\Inv}}) \).

    On the other side we obtain an equivalence \( f_2 : U_2 \simeq {\R}_{\Sh} \)
    by distributing \( \Isolated{(\Blank)} \) over binary sums:
    \begin{align*}
      U_2
      &\DefEq
      \sum_{s : S} \sum_{f : P_1(s) \to T}
        \Isolated{P_0(s)} + \Isolated{\big( \smashoperator{\sum_{p : P_1(s)}} Q(fp) \big)}
        \\
      &\simeq
      \sum_{s : S} \sum_{f : P_1(s) \to T}
        \Isolated{\big( P_0(s) + \smashoperator{\sum_{p : {P_1(s)}}} Q(fp) \big)}
        \\
      &\simeq
        {\R}_{\Sh}
    \end{align*}
    Again, let \( H_2 \DefEq (\MkCont{U_2}{{\R}_{\Ps} \circ f_2}) \).

    Let us now define \( \eta : H_1 \multimap H_2 \).
    As in \autoref{lax-chain-rule}, define the shape map \( \eta_{\Sh} : U_1 \to U_2 \) using \( \SigmaIsolate_{P_1(s),Q(f\shortminus)} \).
    On positions, the equivalence
    \(
      \eta_{\Ps}(u) : H_2^{\Ps}(\eta_{\Sh}(u)) \simeq H_1^{\Ps}(u)
    \)
    is defined by cases, depending on which side of the sum \( u : U_1 \) falls.
    Let \( s : S \), \( f : P_1(s) \to T \).
    In case of \( u \JudgeEq (s, f, \Inl{p_0}) \) for \( p_0 : \Isolated{P_0(s)} \), our goal is to give
    \[
      ( \Isolated{P_0(s)} + B ) \setminus \Inl(p_0)
        \simeq
      ( \Isolated{P_0(s)} \setminus p_0 ) + B
      \quad
      (\text{where } B \DefEq {\textstyle \sum_{p : P_1(s)} Q(fp) }),
    \]
    which is an instance of \autoref{sum-remove-equiv}.

    When \( u \JudgeEq (s, f, \Inr(p_1, q)) \) for some \( p_1 : \Isolated{P_1(s)} \) and \( q : \Isolated{Q(fp)} \),
    we rewrite as follows:
    \begin{align}
      H_2^{\Ps}(\eta_{\Sh}(s , f , \Inr(p_1, q)))
        &\mathrel{\JudgeEq}
          \big( P_0(s) + \smashoperator{\sum_{p : P_1(s)}} Q(f p) \big) \setminus \Inr(p_1, q)
          \notag
          \\
        &\simeq
          P_0(s) + \big( \smashoperator{\sum_{p : P_1(s)}} Q(f p) \big) \setminus (p_1, q)
          \label{binary-chain-rule-pos-equiv-sum-minus}
          \\
        &\simeq
          P_0(s) + \smashoperator{\sum_{p : P_1(s) \setminus p}} Q(f p) + (Q(f\,p_1) \setminus q)
          \label{binary-chain-rule-pos-equiv-sigma-minus}
          \\
        &\mathrel{\JudgeEq}
          H_1^{\Ps}(s, f, \Inr{(p_1, q)})
          \notag
    \end{align}
    In \eqref{binary-chain-rule-pos-equiv-sum-minus}, we move the pair \( (p_1, q) \) to the right of the sum (\autoref{sum-remove-equiv}).
    For \eqref{binary-chain-rule-pos-equiv-sigma-minus},
    we split the \( \Sigma \)-type by applying \autoref{is-equiv-sigma-remove}.
    This is justified since both \( p_1 \) and \( q \) are isolated points,
    and together, the pair \( (p_1, q) \) is isolated in \( \sum_{p : P_1(s)} Q(f p) \) by \autoref{is-isolated-pair}.
  \end{construction}
\end{problem}

Like for unary containers, the chain-rule is strong whenever isolated points distribute over dependent sums:
\begin{proposition}\label{strong-binary-chain-rule-iff-is-equiv-sigma-isolate}
  For all \( F \JudgeEq (\MkCont{S}{P}) : \Cont_2 \) and \( G \JudgeEq (\MkCont{T}{Q}) : \Cont_1 \), the following are equivalent:
  \begin{enumerate}
    \item \( \Chain_{F,G} \) is an equivalence of unary containers.
    \item For all \( s : S \) and \( f : P_1(s) \to T \), \( \SigmaIsolate_{P_1(s), Q(f(\Blank))} \) is an equivalence.
  \end{enumerate}
  \begin{proof}
    The morphism \( \Chain_{F,G} \) is an equivalence if and only if \( \eta_{\Sh} \) in the above construction is an equivalence of types.
    This in turn is exactly the case when the second condition holds.
  \end{proof}
\end{proposition}

\subsubsection{Smallest Fixed-points of Containers}

Assuming we have access to \( \W \)-types, we encode inductive types as smallest fixed-points of the substitution functor:
For all \( F : \Cont_{I+1} \), there is a container \( {\mu F} : \Cont_I \)
such that \( \Subst{F}{\mu F} \CartEquiv {\mu F} \).
The positions at the \( I + 1\)\textsuperscript{st} index describe
occurrences of the recursion variable \( Y \) in an informal fixed-point specification of an inductive type
such as \( \Op{List}(X) = \mu Y.\, 1 + X \times Y \).

Recall that for \( A : \Type \) and \( B : A \to \Type \),
the type \( \W(A, B) \) has the single constructor
\begin{equation*}
  {
    \AxiomC{\( a : A \)}
    \AxiomC{\( f : B(a) \to \W(A, B) \)}
    \BinaryInfC{\( \Sup(a, f) : \W(A, B) \)}
    \DisplayProof
  }
\end{equation*}
Each \( w : \W(A, B) \) is a tree with \( A \)-labeled nodes, and \( B(a) \)-branching subtrees.
A path to some node inside \( w \), labelled by \( C : A \to \Type \),
can be defined as an inductive family \( \bar{\W}_{A,B,C} : \W(A,B) \to \Type \) with two constructors, namely
\begin{center}
  \hspace*{\fill}
  {
    \AxiomC{\( c : C(a) \)}
    \UnaryInfC{\( \Op{top}(c) : \bar{\W}_{A, B, C}(\Sup(a, f)) \)}
    \DisplayProof
  }
  \hfill%
  and
  \hfill%
  {
    \AxiomC{\( b : B(a) \)}
    \AxiomC{\( w : \bar{\W}_{A,B,C}(f(b)) \)}
    \BinaryInfC{\( \Op{below}(b, w) : \bar{\W}_{A, B, C}(\Sup(a, f)) \)}
    \DisplayProof
  }
  \hspace*{\fill}
\end{center}
for all \( a : A \) and \( f : B(a) \to \W(A, B) \).

Together, we can use \( \W \) and \( \bar{\W} \) to define the shapes and position, respectively, of the smallest fixed-point container:
\begin{definition}
  Let \( F \JudgeEq (\MkCont{S}{\vec{P},P}) : \Cont_{I+1} \), define \( {\mu F} \DefEq (\MkCont{S^\mu}{P^\mu}) : \Cont_I \):
  \begin{align*}
    S^\mu &\DefEq \W(S, P) \\
    P^\mu_i & \DefEq \bar{\W}_{S,P,\vec{P}}
    \qedhere
  \end{align*}
\end{definition}

Both \( \W \) and \( \bar{\W} \) can be described by unfolding them by one level
---
the constructors define equivalences
\begin{align*}
  \operatorname{\Op{W-in}} &: \sum_{a : A} (B(a) \to \W(A, B)) \simeq \W(A, B)
    \\
  \operatorname{\Op{\bar{W}-in}}_{a, f} &: C(a) + \sum_{b : B(a)} \bar{\W}_{A,B,C}(f(b)) \simeq \bar{\W}_{A,B,C}(\Sup(a, f))
\end{align*}
for all \( a : A \) and \( f : B(a) \to \W(A, B)\).
These let us derive a fixed-point to substitution:
\begin{problem}
  Define an equivalence of containers \( \Op{in}_F : F[{\mu F}] \CartEquiv {\mu F} \).
  \begin{construction}
    Let \( F \JudgeEq (\MkCont{S}{\vec{P},P}) \).
    Shapes of \( \Subst{F}{\mu F} \) are \( \sum_{s : S} (P(s) \to \W(S, P)) \),
    hence \( \operatorname{\Op{W-in}} \) establishes an equivalence with the shapes of \( \mu F \).
    Similarly, the positions of \( \Subst{F}{\mu F} \) are equivalent to those of \( \mu F \) via \( \operatorname{\Op{\bar{W}-in}} \).
  \end{construction}
\end{problem}

% \begin{example}
%   Binary trees are inductive types such that \( \Op{Tr}(X) = X + X \times \Op{Tr}(X)^2 \).
%   As a container, we encode \( \Op{Tr} \DefEq \mu \Op{T} : \Cont_1 \) as follows:
%   \( \Op{T} = (\MkCont{S}{P}) \) is a binary container with a set of two shapes, \( \{ \Op{leaf}, \Op{branch} \} \),
%   over which there are the following positions for each of the two indices:
%   \begin{align*}
%     P_0(\Op{leaf}) &\DefEq 1  & P_0(\Op{branch}) &\DefEq 1 \\
%     P_1(\Op{leaf}) &\DefEq 0  & P_1(\Op{branch}) &\DefEq 2
%   \end{align*}
%   The positions \( P_0 \) and \( P_1 \) correspond, respectively, to occurrences of the variables
%   \( X \) and \( Y \) in the fixed-point specifications \( \Op{Tr} = \mu Y.\, X + X \times Y^2 \).
% \end{example}

\begin{todo-block}
  We observe the \( \mu \)-rule has the shape of an algebra for a \( \Subst{G}{\Blank} \)-functor,
  given some container \( G \) derived from \( F \).
\end{todo-block}

We define morphisms out of a fixed-point via a non-dependent recursion principle:
\begin{problem}\label{mu-rec}
  Define an induction principle for morphisms out of \( \mu \)-containers.
  That is, a morphism
  \begin{center}
    \hspace*{\fill}
    {
      \AxiomC{\( F : \Cont_{I+1} \)}
      \AxiomC{\( G : \Cont_I \)}
      \AxiomC{\( \alpha : \Cart{F[G]}{G} \)}
      \TrinaryInfC{\( \Op{rec}_{F}(\alpha) : \Cart{\mu F}{G} \)}
      \DisplayProof
    }
    \hspace*{\fill}
  \end{center}
  with a filler of
    \begin{equation}\label{mu-rec-comm}
      \begin{tikzcd}[column sep=huge]
        \Subst{F}{\mu F} & \Subst{F}{G} \\
        {\mu F}          & G
        \ar[from=1-1, to=1-2, -multimap, "\Subst{F}{\Op{rec}_F(\alpha)}"]
        \ar[from=1-1, to=2-1, -multimap, "\Op{in}_F"{swap}]
        \ar[from=2-1, to=2-2, -multimap, "\Op{rec}_F(\alpha)"{swap}]
        \ar[from=1-2, to=2-2, -multimap, "\alpha"]
      \end{tikzcd}
    \end{equation}
    for all \( \alpha : \Cart{\Subst{F}{G}}{G} \).
\end{problem}

\begin{todo-block}
  It should not be hard to show (either do it or check who has done it)
  that \( \mu F, \In_F \) is the initial algebra of the wild endofunctor \( \Subst{F}{\Blank} \),
  in the sense that for all \( \alpha : \Cart{\Subst{F}{G}}{G} \), the type of morphisms \( \alpha^\ast : \Cart{\mu F}{G} \)
  such that \autoref{mu-rec-comm} commutes is contractible.
\end{todo-block}

\begin{problem}[note={Lax \( \mu \)-rule}]\label{mu-rule}
  For \( F : \Cont_2 \), define a cartesian morphism
  \[
    \MuRule_F
      :
    \Cart%
      {
        \mu(
          { \Wk{\Der_0{F}[ {\mu F} ]} }
            \CPlus
          (
            \Wk{\Der_1{F}[ {\mu F} ]}
              \CTimes
            \Proj{1}
          )
        )
      }%
      {\Der(\mu F)}
  \]
\end{problem}
  \begin{construction}
    We define the morphism using the induction principle (\autoref{mu-rec}).
    Our goal is to provide some
    \[
      \alpha : \Cart{G[\Der(\mu F)]}{\Der(\mu F)}
    \]
    where \( G \DefEq { \Wk{\Der_0{F}[ {\mu F} ]} } \CPlus ( \Wk{\Der_1{F}[ {\mu F} ]} \CTimes \Proj{1} ) \).
    First note that \( G \) is a container in two variables,
    and substitution into the second replaces \( \Proj{1} \), i.e.\@ there is an equivalence
    \[
      \gamma_{\Highlight{Y}} :
      G[\Highlight{Y}]
        \mathrel{\CartEquiv}
      { \Der_0{F}[ {\mu F} ] } \CPlus ( {\Der_1{F}[ {\mu F} ]} \CTimes \Highlight{Y} )
    \]
    Equipped with this knowledge, we use the chain rule to define \( \alpha \) as the following composite:
    \begin{equation*}
      \begin{tikzcd}[column sep=large]
        {G[\Highlight{\Der(\mu F)}]}
          &
        {\Der(\mu F)}
          \\
        {
          { \Der_0{F}[ {\mu F} ] }
            \CPlus
          (
            {\Der_1{F}[ {\mu F} ]}
              \CTimes
            \Highlight{\Der(\mu F)}
          )
        }
          &
        \Der(F[\mu F])
        \ar[from=1-1, to=2-1, -multimap, "\gamma_{\Highlight{\Der(\mu F)}}"{swap}, "\sim"]
        \ar[from=2-1, to=2-2, -multimap, "\Op{chain}_{F,{\mu\!F}}"{swap}]
        \ar[from=2-2, to=1-2, -multimap, "\Der(\Op{in}_F)"{swap}, "\sim"]
        \ar[from=1-1, to=1-2, dashed, -multimap, "\alpha"]
      \end{tikzcd}
    \end{equation*}
    Lastly, define \( \MuRule_F \DefEq \Op{rec}_{G}(\alpha) : \Cart{ \mu G }{ \Der{(\mu F)} } \).
  \end{construction}

% Our goal is to derive a fixed-point rule \( \varphi : \CartIso{\mu F^{\prime}}{\Der{\mu F}} \) for some \( F^{\prime} \).
% Assuming we had defined \( \varphi \) by recursion from a \( \Subst{F^{\prime}}{\Blank} \)-algebra,
% Importantly, \( \varphi \) being an equivalence is reflected by recursion:
One advantage of defining the \( \mu \)-rule by recursion is that it highlights the dependency on the chain-rule.
First, observe that recursion reflects equivalence, in the following sense:
\begin{lemma}\label{is-equiv-from-mu-rec}
  Let \( \alpha : \Cart{\Subst{F}{G}}{G} \).
  If \( \Op{rec}_{F}(\alpha) \) is an equivalence, then so is \( \alpha \).
  \begin{proof}
    Substitution \( \Subst{F}{-} \) preserves equivalences,
    so by 3-for-2 for equivalences of containers, \( \alpha \) on the right is an equivalence:
    \[
      \begin{tikzcd}[column sep=huge]
        \Subst{F}{\mu F} & \Subst{F}{G} \\
        {\mu F}          & G
        \ar[from=1-1, to=1-2, multimap-multimap, "\Subst{F}{\Op{rec}_F(\alpha)}"]
        \ar[from=1-1, to=2-1, multimap-multimap, "\Op{in}_F"{swap}]
        \ar[from=2-1, to=2-2, multimap-multimap, "\Op{rec}_F(\alpha)"{swap}]
        \ar[from=1-2, to=2-2, -multimap, "\alpha"]
      \end{tikzcd}
      \qedhere
    \]
  \end{proof}
\end{lemma}

Thus, a strong \( \mu \)-rule for some container necessarily implies a strong rule between \( F \) and \( \mu F \):
\begin{proposition}\label{strong-chain-rule-from-strong-mu-rule}
  Let \( F : \Cont_{I + 1} \).
  If \( \MuRule_F \) is an equivalence, then so is \( \Op{chain}_{F,{\mu F}} \).
  \begin{proof}
    Assume that \( \MuRule_F \) is an equivalence.
    In \autoref{mu-rule}, \( \MuRule_F \) is defined by recursion from some \( \alpha \),
    hence \( \alpha \) is an equivalence by \autoref{is-equiv-from-mu-rec}.
    But \( \alpha \) is just \( \Op{chain}_{F, {\mu F}} \) wedged between equivalences,
    so the latter is an equivalence.
  \end{proof}
\end{proposition}

With a bit of creativity we can show that this is not only a necessary,
but also a sufficient condition:
\begin{theorem}\label{strong-mu-rule-iff-strong-chain-rule}
  For any container \( F : \Cont_{I + 1} \), the following are equivalent:
  \begin{enumerate}
    \item \( \MuRule_F \) is an equivalence.
    \item \( \Op{chain}_{F,{\mu F}} \) is an equivalence.
  \end{enumerate}
  \begin{proof}
    One direction is exactly \autoref{strong-chain-rule-from-strong-mu-rule}.
    Let us sketch the other direction; for details, see the formalization.

    Assume \( \Op{chain}_{F,{\mu F}} \) to be an equivalence.
    By \autoref{strong-binary-chain-rule-iff-is-equiv-sigma-isolate}, this is equivalent to isolated pairs
    \( (p_1, \bar{w}) \)
    having isolated components \( p_1 : P_1(s) \) and \( \bar{w} : \bar{\W}(f{p_1}) \),
    for all \( s : S \) and \( f : P_1(s) \to \W_{\!S}(P_1) \).
    This property is exactly what is needed to define a putative inverse on shapes,
    i.e.\@ a map \( \Der{(\mu F)}_{\Sh} \to {(\mu G)}_{\Sh} \).
    Using this map, we then prove that all fibers of the (forward) shape map are inhabited.
    It remains to show that these fibers are propositions:
    By a combination of \( \W \)-induction and \( \Graft \)-induction,
    we see that each fiber is equivalent to a disjoint sum of fibers over the constructors \( \Op{top} \) and \( \Op{below} \).
    But these maps are embeddings, hence any of their fibers are propositions.
  \end{proof}
\end{theorem}

With this at hand, we can give a proof that the \( \mu \)-rule is strong for discrete containers
that factors entirely through the properties of the chain-rule:
\begin{corollary}
  If \( F : \Cont_{I + 1} \) is discrete, then \( \MuRule_F \) is an equivalence.
  \begin{proof}
    By the previous theorem, it suffices to show that \( \Chain_{F,\mu F} \) is an equivalence.
    But \( \mu F \) is discrete, and the chain-rule between discrete containers is strong (\autoref{discrete-strong-chain-rule}).
  \end{proof}
\end{corollary}

\begin{todo-block}
  Is a globally strong \( \mu \)-rule a constructive taboo the same way that a global chain-rule is?
  That is not so obvious.

  First, note that strong \( \MuRule_F \) implies (by \autoref{strong-binary-chain-rule-iff-is-equiv-sigma-isolate}) that
  for every \( s : S \) and \( f : P_1(S) \to \W(S, P_1) \), \( \SigmaIsolate \) defines an equivalence
  \[
    \sum_{p : \Isolated{P_1(s)}} \Isolated{\bar{\W}_{P_0}(f(p))} \simeq 
      \Isolated{\big(
        \sum_{p : P_1(s)} \bar{\W}_{P_0}(f(p))
      \big)}
  \]
  Is this enough to derive that either types of positions, \( P_0 \) or \( P_1 \), are discrete?
  We get:
  \[
      \Isolated{P_0(s)}
        +
      \Isolated{\big(
        \sum_{p : P_1(s)} \bar{\W}_{P_0}(f(p))
      \big)}
      \simeq
      \Isolated{\bar{\W}_{P_0}(\Sup(s, f))}
  \]
\end{todo-block}

\begin{todo-block}
  Sketch a (dis)proof that \( \Chain_{F,\mu F} \) is an equivalence:
  If \( \alpha : \Cart{ \Subst{G}{\Der(\mu F)} }{ \Der(\mu F) } \) is an initial \( \Subst{G}{\Blank} \)-algebra,
  then it is --- by Lambek's Lemma --- an equivalence.
  In this case, \( \Chain_{F, \mu F} \) would be invertible.
\end{todo-block}


\section*{Conclusion}

\begin{itemize}
  \item
    In \cite{AbbottEtAl2003DerivativesContainers}, \citeauthor{AbbottEtAl2003DerivativesContainers} consider the general adjunction \( \Blank \CTimes H \dashv [H, \Blank] \),
    perhaps we can make more progress towards that? By iterating ours, it should be possible to derive \( \Blank \CTimes \Op{K}(n) \dashv \Der^n \).
    For combinatorial species, this is enough to derive the fully general adjunction, since (1) the \( \Op{K}(n) \) are exactly the representables
    \( \Op{y}(n) \), (2) every species is a colimit of some \( \Op{y}(n_i) \)'s, and (3) the product (in this case, Day-convolution) is co-continuous.
  \item
    Say something about largest fixed points. Most proofs should generalize to arbitrary fixed points;
    point out which do.
\end{itemize}


\printbibliography

\end{document}
