\documentclass{article}

\usepackage{amsmath}
\usepackage{amssymb}
\usepackage{mathtools}
\usepackage{csquotes}
\usepackage{tikz-cd}
\usepackage{keytheorems}
\usepackage{todonotes}
\usepackage{hyperref}

\hypersetup{
  colorlinks=true,
  % linkcolor=cyan,
}

\newkeytheoremstyle{proof}{
  inherit-style=definition,
  qed=$\lrcorner$,
  numbered=false,
}
\newkeytheorem{theorem}[style=plain,parent=section]
\newkeytheorem{
  proposition,
  lemma,
  corollary,
  problem,
}[
  style=plain,
  sibling=theorem,
]
\newkeytheorem{definition}[style=definition,sibling=theorem]
\newkeytheorem{construction}[style=proof,sibling=theorem]
\newkeytheorem{remark}[style=remark,sibling=theorem]

\title{Derivates of Containers without Decidable Equality}
\author{Philipp Joram}
\date{\today}

\DeclareMathSymbol{\shortminus}{\mathbin}{AMSa}{"39}

\newcommand*{\Op}[1]{\mathsf{#1}}
\DeclareMathOperator{\Type}{\mathsf{Type}}
\DeclareMathOperator{\Dec}{\mathsf{Dec}}
\DeclareMathOperator{\DefEq}{\mathrel{\coloneq}}
\DeclareMathOperator{\JudgeEq}{\mathrel{\doteq}}
\NewDocumentCommand{\Inv}{}{{\shortminus 1}}

\DeclareMathOperator{\SigmaRemove}{\Op{\Sigma-remove}}
\DeclareMathOperator{\SigmaIsolate}{\Op{\Sigma-isolate}}

\newcommand*{\Isolated}[1]{#1^{\circ}}

\newcommand*{\MkCont}[2]{#1\mathbin{\triangleleft}#2}
\DeclareMathOperator{\Sh}{\mathsf{Sh}}
\DeclareMathOperator{\Ps}{\mathsf{Ps}}

\DeclareMathOperator{\CTimes}{\mathbin{\otimes}}
\DeclareMathOperator{\CPlus}{\mathbin{\oplus}}
\newcommand*{\Subst}[2]{#1[#2]}

\NewDocumentCommand{\SimMultiMap}{}{%
  \mathrel{%
    \vbox{%
      \offinterlineskip%
      \mathsurround=0pt
      \ialign{%
        \hfil##\hfil\cr
        \normalfont\scalebox{0.9}{\kern-.33ex{$\sim$}}\cr
        \noalign{\kern-.25ex}
        $\scriptstyle\multimap$\cr%
      }%
    }%
  }%
}

\newcommand*{\Cart}[2]{#1\mathrel{\multimap}#2}
% \newcommand*{\CartIso}[2]{#1\SimMultiMap#2}
\DeclareMathOperator{\CartEquiv}{\SimMultiMap}
\newcommand*{\CartIso}[2]{#1\cong#2}

\DeclareMathOperator{\Cont}{\mathsf{Cont}}
\DeclareMathOperator{\ContCart}{\Cont^{\multimap}}

\NewDocumentCommand{\Der}{}{\partial}
\NewDocumentCommand{\Id}{}{\mathsf{Id}}

\DeclareMathOperator{\Inl}{\mathsf{inl}}
\DeclareMathOperator{\Inr}{\mathsf{inr}}

\NewDocumentCommand{\Maybe}{m}{#1 \otimes \Id}
\DeclareMathOperator{\Nothing}{\mathsf{nothing}}
\DeclareMathOperator{\Just}{\mathsf{just}}

\makeatletter
\newcommand{\proofsubparagraph}{%
  \@startsection{subparagraph}%
  {5}%
  {\z@}%
  {3.25ex \@plus1ex \@minus .2ex}%
  {-1em}%
  {\sffamily\normalsize\bfseries}%
}
\makeatother

\begin{document}

\maketitle

\begin{abstract}
  We introduce a derivative operation for containers whose positions lack decidable equality.
\end{abstract}

\section{Isolated Points and How to Remove Them}

This section recalls the notion of isolated points of a type.
This definition is not new\todo{figure out how to properly cite \texttt{TypeTopology}, or whatever primary source}.
We are going to prove some properties that will be applied on \autoref{derivatives}
to establish properties of derivatives of containers:
Many of the laws of derivatives -- after some \emph{type-yoga} -- reduce directly to instances of statements in this section.

\begin{definition}
  A point \( a : A \) is \emph{isolated} if \( a = b \) is decidable for all \( b : A \).
  We denote by \( \Isolated{A} \) the subtype of isolated points, that is
  \(
    \Isolated{A} \DefEq \sum_{a : A} \prod_{b : B} \Dec(a = b)
  \).
\end{definition}

Asking for a point to be isolated trivializes the path spaces around it:
\begin{lemma}\label{is-prop-isolated-path}
  If \( a : A \) is isolated, then \( a = b \) is a proposition for all \( b : A \).
  \begin{proof}
    \todo[inline]{This follows from a \enquote{local} version of \emph{Hedberg's Lemma}.}
  \end{proof}
\end{lemma}

\begin{corollary}\label{is-prop-isolated-dec-path}
  If \( a : A \) is isolated, then \( \Dec{(a = b)} \) is a proposition for all \( b : A \).
  \begin{proof}
    For any type \( X \), \( \Dec(X) \) is a proposition whenever \( X \) is.
  \end{proof}
\end{corollary}

\begin{corollary}\label{is-prop-is-isolated}
  Being an isolated point is a proposition.
\end{corollary}

\begin{proposition}\label{is-set-isolated}
  The type of isolated points \( \Isolated{A} \) is a set for any type \( A \).
\end{proposition}

\begin{proposition}\label{is-equiv-forget-isolated-iff-discrete}
  The map \( \Op{fst} : \Isolated{A} \to A \) is an equivalence if and only if \( A \) is discrete.
\end{proposition}

Being an isolated point is stable under equivalence:
\begin{lemma}\label{is-isolated-respect-equiv}
  If \( e : A \simeq B \),
  then \( a : A \) is isolated if and only iff \( e(a) : B \) is isolated.
  We write \( \Isolated{e} : \Isolated{A} \simeq \Isolated{B} \) for the induced equivalence.
\end{lemma}

If a map \( f : A \to B \) is not an embedding,
but still behaves \enquote{nicely} on path spaces, we can deduce its behavior on isolated points.
Recall that \( f \) is an embedding if \( \Op{cong}_f : x = y \to f(x) = f(y) \) is an equivalence,
written \( f : A \hookrightarrow B \).
Such a map reflects isolated points:

\begin{proposition}[note={Embeddings reflect isolated points}]\label{embedding-reflect-isolated}
  Let \( f : A \hookrightarrow B \) and \( a : A \).
  If \( f(a) \) is isolated in \( B \), then \( a \) is isolated in \( A \).
  \begin{proof}
    For all \( a^\prime : A \), we need to decide \( a = a^\prime \).
    By assumption, we can decide whether \( f(a) = f(a^\prime) \) or not.
    If \( f(a) \neq f(a^\prime) \), then necessarily \( a \neq a^\prime \).
    If \( f(a) = f(a^\prime) \), we get \( a = a^\prime \) since \( f \) is an embedding, which we can cancel.
  \end{proof}
\end{proposition}

From this we can deduce that the canonical embeddings \( \Inl : A \hookrightarrow A + B \) and \( \Inr : B \hookrightarrow A + B \)
both reflect and create isolated points:

\begin{proposition}\label{sum-embeddings-respect-isolated}
  Let \( A, B : \Type \).
  A point \( a : A \) is isolated if and only if \( \Inl{(a)} : A + B \) is isolated;
  similarly for \( b : B \) and \( \Inr{(b)} : A + B \).
  \begin{proof}
    Let \( a : A \). In the forward direction, assume that \( a \) is isolated.
    We need to show that for any \( x : A + B \), the type \( \Inl{a} = x \) is decidable.
    Consider the case of \( x \JudgeEq \Inl{a^{\prime}} \).
    We know that \( \Inl \) is an embedding, and as such there is an equivalence
    of path spaces \( (\Inl{a} = \Inl{a^\prime}) \simeq (a = a^\prime) \).
    But \( (a = a^\prime) \) is decidable by assumption,
    hence \( (\Inl{a} = \Inl{a^\prime}) \) is decidable.
    In case \( x \JudgeEq \Inr{b} \), the type \( \Inl{a} = \Inr{b} \) is empty, hence decidable.
    For the converse, apply \autoref{embedding-reflect-isolated}:
    The map \( \Inl \) is an embedding, and as such reflects isolated points.
  \end{proof}
\end{proposition}

We see that isolated points distribute over sums:
\begin{problem}
  Construct an equivalence \( \Isolated{(A + B)} \simeq \Isolated{A} + \Isolated{B} \).
  \begin{construction}
    Define the obvious forward- and backward maps by case analysis,
    and prove that points are isolated using \autoref{sum-embeddings-respect-isolated}.
    That these maps are mutually inverse follows since being isolated is a proposition (\autoref{is-prop-is-isolated}).
  \end{construction}
\end{problem}

From this we immediately see that \( \Nothing \DefEq \Inr{(\bullet)} : A + 1 \) is an isolated point, since \( \bullet : 1 \)
is trivially isolated:

\begin{corollary}\label{is-isolated-nothing}
  The point \( \Nothing : A + 1 \) is isolated for any type \( A \),
  and there is an equivalence \( \Isolated{(A + 1)} \simeq \Isolated{A} + 1 \).
\end{corollary}

Later on (\autoref{lax-chain-rule}),
it will become necessary to investigate whether isolated points distribute over other type formers,
in particular dependent sums.
\begin{proposition}\label{is-isolated-pair}
  Let \( A : \Type \) and \( B : A \to \Type \) with points \( a_0 : A \) and \( b_0 : B(a_0) \).
  If both \( a_0 \) and \( b_0 \) are isolated, then \( (a_0 , b_0) \) is isolated in \( \sum_{a : A} B(a) \).
  \begin{proof}
    Let \( a : A \), \( b : B(a) \). Our goal is to decide whether \( (a_0 , b_0) = (a, b) \) or not.
    By extensionality of path types of dependent sums
    it suffices to decide the equivalent type \( \sum_{p : a_0 = a} b_0 = \Op{subst}_B(p^{\Inv}, b_0) \).
    If \( a_0 \neq a \), then this type is empty.
    Otherwise, we have some \( p : a_0 = a \), and the type is inhabited or empty
    depending on whether \( b_0 = \Op{subst}_B(p^{\Inv}, b_0) \) or not.
  \end{proof}
\end{proposition}

\begin{problem}\label{sigma-isolate}
  Given \( A : \Type \) and \( B : A \to \Type \),
  define a map
  \[
    \SigmaIsolate_{A,B} :
      \sum\nolimits_{a_0 : \Isolated{A}} \Isolated{B(a_0)}
        \to
      \Isolated{%
        \big(
          \sum\nolimits_{a : A} B(a)
        \big)
      }
  \]
  \begin{construction}
    Define \( \SigmaIsolate((a, \mathunderscore), (b, \mathunderscore)) \DefEq ((a, b), \mathunderscore) \);
    the pair \( (a, b) \) is isolated by \autoref{is-isolated-pair}.
  \end{construction}
\end{problem}

\begin{proposition}\label{discrete-is-equiv-sigma-isolated}
  If \( A : \Type \) and \( B : A \to \Type \) are (pointwise) discrete types,
  then \( \SigmaIsolate_{A,B} \) is an equivalence.
  \begin{proof}
    For discrete \( A \) and \( B \)
    \autoref{is-equiv-forget-isolated-iff-discrete} yields a commutative triangle
    \[
      \begin{tikzcd}[column sep=small]
          \sum_{a_0 : \Isolated{A}} \Isolated{B(a_0)}
            &
          %
            &
          \Isolated{\big(\sum_{a : A} B(a)\big)}
        \\
          %
            &
          {\sum_{a : A} B(a)}
            &
          %
        \ar[from=1-1, to=1-3, "\SigmaIsolate"]
        \ar[from=1-1, to=2-2, "\simeq"{description}]
        \ar[from=2-2, to=1-3, "\simeq"{description}]
      \end{tikzcd}
    \]
    hence \( \SigmaIsolate \) is an equivalence by the \emph{2-out-of-3}-property.
  \end{proof}
\end{proposition}

\begin{lemma}\label{is-isolated-fst-discrete}
  A type \( A : \Type \) is discrete if the following proposition holds:
  \[
    % \operatorname{\Op{\Sigma-isolate-fst}}_A :
     \prod_{B : A \to \Type}
      \prod_{a : A}
      \prod_{b : B(a)}
      \operatorname{\Op{isIsolated}}{(a, b)}
      \to
      \operatorname{\Op{isIsolated}} a
  \]
  \begin{proof}
    It suffices to show that all points \( a_0 : A \) are isolated.
    The type of singletons \( \sum_{a : A} a_0 = a \)
    is contractible, hence all of its points are isolated.
    We apply the assumption to \( B(a) \DefEq {(a_0 = a)} \),
    \( a \DefEq a_0 \) and \( b \DefEq \Op{refl}_{a_0} \),
    from which we conclude that \( a_0 \) is isolated.
  \end{proof}
\end{lemma}

\begin{proposition}\label{is-equiv-sigma-isolate-discrete}
  Let \( A : \Type \).
  If \( \SigmaIsolate_{A,B} \) is an equivalence for all families \( B : A \to \Type \),
  then \( A \) is discrete.
  \begin{proof}
    We prove discreteness of \( A \) by appealing to \autoref{is-isolated-fst-discrete}.
    Let \( B : A \to \Type \),
    and denote by \( u : \Isolated{\big(\sum_{a : A} B(a)\big)} \to \sum_{a : \Isolated{A}} \Isolated{B(a)} \)
    the assumed inverse to \( \SigmaIsolate_{A,B} \).
    For any \( a_0 : A \), \( b_0 : B(a_0) \), and \( p_0 : \Op{isIsolated}{(a_0, b_0)} \),
    it remains to show that \( a_0 \) is isolated.
    Let \( y \DefEq u((a_0 , b_0) , p_0) \),
    where \( y \JudgeEq ((a, p_a), (b, p_b)) \)
    and \( p_a \) is proof that \( a : A \) is isolated.
    We are done if we can show that \( a_0 = a \).
    We do so by noticing that both \( \SigmaIsolate(y) \JudgeEq ((a, b), \mathunderscore) \) and \( ((a_0, b_0), p_0) \)
    lie in \( \Op{fiber}_u(y) \), which is necessarily contractible,
    hence \( a = a_0 \).
  \end{proof}
\end{proposition}

\subsection{Removing points}


For any type \( A \) and point \( a : A \) we define the subtype of \enquote{\( A \) with \( a \) removed}
to be \( A \setminus a \DefEq \sum_{b : A} a \neq b \).
Whenever \( a \) is isolated in \( A \), a number of useful properties hold:

\begin{problem}\label{isolated-minus-plus-equiv}
  For all \( a_0 : \Isolated{A} \), define an equivalence \( (A \setminus a_0) + 1 \simeq A \).
  \begin{construction}
    We define the equivalence from mutually inverse maps.
    In the forward direction, we define \( f(\Just{(a , \mathunderscore)}) \DefEq a \) and \( f(\Nothing) \DefEq a_0 \).
    In the other direction, we define \( \Op{r} : \prod_{a : A} \Dec{a_0 = a} \to (A \setminus a_0) + 1 \)
    by
    \[
      \Op{r}(a, d) \DefEq
      \begin{cases}
        \Nothing & \text{if}\ d \JudgeEq \Op{yes}(\mathunderscore : a_0 = a) \\
        \Just{(a, h)} & \text{if}\ d \JudgeEq \Op{no}(h : a_0 \neq a)
      \end{cases}
    \]
    For \( i_0 : \Op{isIsolated}(a_0) \), let \( g(a) \DefEq \Op{r}(a, i_0(a)) \).
    By \autoref{is-prop-isolated-dec-path}, \( i_0(a) : \Dec(a_0 = a) \) is a proposition for all \( a : A \);
    we use this to ensure that \( g \) computes correctly:
    \begin{align*}
      g(a_0) &\JudgeEq \Op{r}(a_0, i_0(a_0)) \overset{\scriptstyle!}{=} \Op{r}(a_0, \Op{yes}(\Op{refl})) \JudgeEq \Nothing \\
      g(a)   &\JudgeEq \Op{r}(a, i_0(a)) \overset{\scriptstyle!}{=} \Op{r}(a, \Op{no}(h)) \JudgeEq \Just{(a, h)}
        \quad \forall (a, h) : A \setminus a_0
    \end{align*}
    From this we get that \( f \circ g = \Op{id} \) and \( g \circ f = \Op{id} \).
  \end{construction}
\end{problem}

The above is only guaranteed to be an equivalence when the point being removed is isolated:
Consider for example the circle \( S^1 \) with \( \mathsf{base} : S^1 \).
Then \( S^1 \setminus \mathsf{base} \simeq 0 \),
hence \( (S^1 \setminus \mathsf{base}) + 1 \not\simeq S^1 \):
removing \( \mathsf{base} \) removes the entire circle, which is \emph{not} contractible.

\begin{problem}\label{maybe-minus-nothing-equiv}
  Define an equivalence \( (A + 1) \setminus \Nothing \simeq A \).
  \begin{construction}
    Define \( f : (A + 1) \setminus \Nothing \to A \) as \( f(\Just{a} , \mathunderscore) \DefEq a \);
    the case \( f(\Nothing, h : \Nothing \neq \Nothing) \) is absurd.
    It is straightforward to show that the fibers of \( f \) are contractible.
  \end{construction}
\end{problem}

\begin{remark}
  Alternatively, apply \autoref{isolated-minus-plus-equiv} to \( \Nothing : \Isolated{A + 1} \),
  obtain an equivalence
  \( e : ((A + 1) \setminus \Nothing) + 1 \simeq A + 1 \),
  notice that \( e \) fixes the right summand,
  and cancel to \( (A + 1) \setminus \Nothing \simeq A \).
\end{remark}

\begin{problem}\label{sigma-remove}
  Let \( A : \Type \) and \( B : A \to \Type \)
  with points \( a_0 : A \) and \( b_0 : B(a_0) \),
  and assume \( p : \Op{isProp}(a_0 = a_0) \).
  There is a map
  \[
    \operatorname{\Op{\Sigma-remove}}_p :
    \big(\smashoperator{\sum_{a : A \setminus a_0}} B(a)\big) + \big(B(a_0) \setminus b_0\big)
      \to
    \big(\sum_{a : A} B(a) \big) \setminus (a_0 , b_0)
  \]
  \begin{construction}
    We define \( \operatorname{\Op{\Sigma-remove}}_p(x) \) by cases.
    Let
    \[
      \SigmaRemove_p(\Inl{(a, h_a, b)}) \DefEq ((a, b) , h^\prime_a),
    \]
    where \( h^\prime_a : (a_0 , b_0) = (a, b) \xrightarrow{\Op{cong}_{\Op{fst}}} a_0 = a \xrightarrow{h_a} \bot \).
    In the other case, let
    \[
      \SigmaRemove_p(\Inr{(b, h_b)}) \DefEq ((a_0, b) , h^\prime_b),
    \]
    and show \( h^\prime_b : (a_0 , b_0) \neq (a_0 , b) \) as follows:
    Assume to the contrary that \( p_b : (a_0 , b_0) = (a_0 , b) \).
    From this we obtain (dependent) paths \( p_b^1 : a_0 = a_0 \) and \( p_b^2 : b_0 =_{p_b^1} b \).
    Since \( a_0 = a_0 \) is a proposition, we know that \( p_b^1 = \Op{refl} \),
    hence \( b_0 = b \).
    This is contradictory since we are given \( h_b : b_0 \neq b \).
  \end{construction}
\end{problem}

\begin{proposition}\label{is-equiv-sigma-remove}
  Let \( A : \Type \), \( B : A \to \Type \) with \( a_0 : A \) and \( b_0 : B(a_0) \).
  If \( a_0 \) is an isolated point of \( A \), then \( \SigmaRemove \) of \autoref{sigma-remove}
  is an equivalence.
  \begin{proof}
    First note that \( a_0 = a_0 \) is a proposition by \autoref{is-prop-isolated-path},
    thus the map is well-defined.
    We construct an inverse
    \[
      \SigmaRemove^{\Inv}
        :
      \Big(\sum_{a : A} B(a) \Big) \setminus (a_0 , b_0)
        \to
      \Big(\smashoperator{\sum_{a : A \setminus a_0}} B(a)\Big) + \big(B(a_0) \setminus b_0\big)
    \]
    as follows:
    Introduce \( a : A \), \( b : B(a) \) and \( h : (a_0 , b_0) \neq (a , b) \),
    then decide whether \( a_0 = a \) or not.
    If \( p : a_0 = a \), we map to \( \Inr{(\Op{subst}_B(p, a), h^\prime)} \),
    where \( h^\prime : b_0 \neq \Op{subst}_B(p, a) \),
    which we conclude from \( h \) and \( p \).
    In case that \( h : a_0 \neq a \) we map to \( \Inl{((a, h), b)} \) directly.
  \end{proof}
\end{proposition}

\subsection{Grafting}

\todo[inline]{Yes, \emph{grafting} is the technical term that Abbott {et al.\@} use.}
\todo[inline]{This is an recursion principle for isolated points.}

\begin{problem}
  For types \( A \) and \( B \), construct a function
  \[
    \Op{graft} : \prod_{\Isolated{a} : \Isolated{A}}
      \big((A \setminus \Isolated{a} \to B) \times B \big)
        \to
      (A \to B)
  \]
  \begin{construction}
    Let \( \Isolated{a} : \Isolated{A} \), \( f : A \setminus \Isolated{a} \to B \) and \( b_0 : B \).
    Decide equality with \( \Isolated{a} \) to define \( \Op{graft}_{\Isolated{a}}(f, b_0) : A \to B \) as follows:
    \[
      \Op{graft}_{\Isolated{a}}(f, b_0) \DefEq
      \lambda a.\,
      \begin{cases}
        f(a, h) & \text{if}\ (h : \Isolated{a} \neq a) \\
        b_0 & \text{otherwise}
      \end{cases}
      \qedhere
    \]
  \end{construction}
\end{problem}
\begin{proposition}\label{graft-equiv}
  For all \( \Isolated{a} : \Isolated{A} \),
  \(
    \Op{graft}_{\Isolated{a}} :
    \big((A \setminus \Isolated{a} \to B) \times B \big)
      \simeq
    (A \to B)
  \)
  is an equivalence of types.
\end{proposition}

\section{Containers}

\begin{definition}
  A \emph{container} \( (\MkCont{S}{P}) \) consists of \emph{shapes} \( S : \Type \) and
  a family \( P : S \to \Type \) of \emph{positions}.
  We access shapes and positions via postfix projections
  \( (\MkCont{S}{P})_{\Sh} \DefEq S \) and \( (\MkCont{S}{P})_{\Ps} \DefEq P \).
\end{definition}

\begin{definition}
  Let \( F \JudgeEq (\MkCont{S}{P}) \) and \( G \JudgeEq (\MkCont{T}{Q}) \).
  The type of \emph{cartesian morphisms} between \( F \) and \( G \) is
  \[
    \Cart{F}{G} \DefEq \sum_{f : S \to T} \prod_{s : S} Q_{fs} \simeq P_s
  \]
  A cartesian morphism \( (f, u) : \Cart{F}{G} \) is an \emph{equivalence} of containers
  if \( f \) is an equivalence of types.
\end{definition}

\begin{definition}
  Containers and cartesian morphisms form a wild category \( \ContCart \).
\end{definition}

\section{Derivatives}\label{derivatives}

\begin{definition}
  The derivative of a container \( \Der{(\MkCont{S}{P})} \DefEq (\MkCont{S'}{P'}) \)
  has shapes \( S' \DefEq \sum_{s : S} \Isolated{P_s} \) and positions \( P'(s , p) \DefEq P_s \setminus p \).
\end{definition}

\begin{definition}
  A container \( (\MkCont{S}{P}) \) is \( (n,k) \)-truncated if \( S \) is \( n \)-truncated,
  and \( P_s \) is \( k \)-truncated for all \( s : S \).
\end{definition}

\begin{corollary}
  For \( n \geq 0 \) and \( k \geq -1 \), the derivative of an \( (n, k) \)-truncated container is \( (n, k) \)-truncated.
\end{corollary}
\begin{proof}
  Let \( (\MkCont{S}{P}) \) an \( (n, k) \)-truncated container.
  By \autoref{is-set-isolated} \( \Isolated{P_s} \) is a 0-truncated type and \( S \) is \( n \)-truncated,
  thus \( \Der{(\MkCont{S}{P})}_{\Sh} \JudgeEq \sum_{s : S} \Isolated{P_s} \) is \( n \)-truncated.
  Positions are \( k \)-types since \( P_s \setminus p \) embeds into \( P_s \).
\end{proof}

The derivative acts on cartesian morphisms.
\begin{problem}
  Define a wild endofunctor \( \Der : \ContCart \to \ContCart \).
  \begin{construction}
    For any \( (f, u) : \Cart{(\MkCont{S}{P})}{(\MkCont{T}{Q})} \),
    there is a canonical morphism \( \Der{(f, u)} \DefEq (f' , u') : \Cart{\Der{(\MkCont{S}{P})}}{\Der{(\MkCont{T}{Q})}} \)
    obtained as follows:
    On shapes, the map \( f' : \sum_{s : S} \Isolated{P_s} \to \sum_{t : T} \Isolated{G_t} \)
    applies \( f \) to the first component and \( \Isolated{(u_s^\Inv)} : \Isolated{P_s} \simeq \Isolated{G_{fs}} \) to the second.
    On positions, \( u'_{s,p} : G_{fs} \setminus u_s^\Inv(p) \simeq F_s \setminus p \) is obtained from \( u_s \), respecting the removed point
    (cf.\@ \autoref{is-isolated-respect-equiv}).
  \end{construction}
\end{problem}


\subsection{Linear adjunction}

\begin{problem}
  Define an adjunction \( (\eta, \varepsilon) : \Maybe{-} \vdash \Der{(-)} \).
  \begin{construction}
    Let \( F \JudgeEq (\MkCont{S}{P}) \) and define \( \eta_F : \Cart{F}{\Der{(\Maybe{F})}} \).
    On shapes, \( {\eta_F^{\Sh}} : S \to \sum_{(s, \mathunderscore) : S \times 1} \Isolated{(P_s + 1)} \)
    sends \( s \) to \( (s, \bullet) \) and \( \mathsf{nothing} \);
    the latter is isolated by \autoref{is-isolated-nothing}.
    On positions, define \( \eta_F^{\Ps} : \prod_{s : S} (P_s + 1) \setminus \Nothing \simeq P_s \)
    as in \autoref{maybe-minus-nothing-equiv}.

    Let \( G \JudgeEq (\MkCont{T}{Q}) \); define the counit \( \varepsilon_G : \Cart{\Maybe{(\Der{G})}}{G} \) as follows:
    on shapes, \( \varepsilon_G^{\Sh} : \sum_{t : T} \Isolated{Q_t} \times 1 \to T \) is the first projection.
    On positions the equivalence
    \(
      \varepsilon_G^{\Ps}(t , q) : Q_t \simeq (Q_t \setminus q) + 1
    \)
    is given by \autoref{isolated-minus-plus-equiv} for all \( t : T \) and \( q : \Isolated{Q_t} \).

    It remains to prove the zigzag equations
    \begin{align*}
      &
      \begin{tikzcd}[ampersand replacement=\&, column sep=small]
        \Maybe{F} \& \& \Maybe{F} \\
                  \& \Maybe{(\Der{(\Maybe{F})})} \& %
        \ar[from=1-1, to=1-3, "\Op{id}"]
        \ar[from=1-1, to=2-2, "\Maybe{{\eta_F}}"{'}]
        \ar[from=2-2, to=1-3, "\varepsilon_{\Maybe{F}}"{'}]
      \end{tikzcd}
        &
      &
      \begin{tikzcd}[ampersand replacement=\&, column sep=small]
                  \& \Der{(\Maybe{\Der{G}})} \& \\
        \Der{G} \& \& \Der{G} %
        \ar[from=2-1, to=2-3, "\Op{id}"]
        \ar[from=2-1, to=1-2, "\eta_{\Der{G}}"]
        \ar[from=1-2, to=2-3, "\Der{(\varepsilon_{G})}"]
      \end{tikzcd}
    \end{align*}
    After applying the necessary extensionality principles
    for functions, equivalences and sum types,
    these hold almost definitionally.
    Only some proofs of isolation and removal need to be compared up to propositional equality.
  \end{construction}
\end{problem}

\subsection{Laws of Derivates}

\begin{proposition}
  For any proposition \( P \), there is an equivalence of containers
  \[
    \CartIso{ \Der{(\MkCont{1}{P})} }{ (\MkCont{P}{0}) }
  \]
  \begin{proof}
    \begin{align*}
      \Der{(\MkCont{1}{P})}
        &\CartEquiv (\MkCont{((\mathunderscore, p) : 1 \times P)}{P \setminus p})
          \\
        &\CartEquiv (\MkCont{(p : P)}{P \setminus p})
          \\
        &\CartEquiv (\MkCont{P}{0})
    \end{align*}
  \end{proof}
\end{proposition}

\begin{corollary}
  \[
    \CartIso{\Der{(\Id)}}{\operatorname{\mathsf{K}}(1)}
  \]
\end{corollary}


\begin{proposition}
  \begin{itemize}
    \item \( \CartIso{\Der{(\operatorname{\mathsf{K}}(S))}}{\operatorname{\mathsf{K}}(0)} \)
    \item \( \CartIso{\Der{(F \oplus G)}}{\Der{F} \oplus \Der{G}} \)
    \item \( \CartIso{\Der{(F \otimes G)}}{(\Der{F} \otimes G) \oplus (F \otimes \Der{G})} \)
  \end{itemize}
\end{proposition}

\begin{problem}[note={The lax chain rule}]\label{lax-chain-rule}
  For any two containers \( F, G \), define a morphism
  \[
    \Op{chain}_{F,G} :
    \Cart%
      { \Subst{(\Der{F})}{G} \CTimes \Der{G} }%
      {\Der{(\Subst{F}{G})}}
  \]
\end{problem}
\begin{construction}[note={for \autoref{lax-chain-rule}}]
  Let \( F \JudgeEq (\MkCont{S}{P}) \) and \( G \JudgeEq (\MkCont{T}{Q}) \).
  As usual, we have to construct a map on shapes and an equivalence of positions.
  On shapes, our goal is a map
  \[
    \big(\sum\nolimits_{(s, p) : \sum_{s : S} \Isolated{P_s}} (P_s \setminus p \to T) \big)
      \times
    \sum_{t : T} \Isolated{Q_t}
      \to
    \Big(
      \sum\nolimits_{(s, f) : \sum_{s : S} (P_s \to T)} \Isolated{\big( \sum_{p : P_s} Q_{fp} \big)}
    \Big)
  \]
  Let us first reshape the left side by some equivalences.
  By re-associating the sums, we obtain
  \begin{align*}
      %
    &\mathrel{\hphantom{\simeq}}
      \big(\sum\nolimits_{(s, p) : \sum_{s : S} \Isolated{P_s}} P_s \setminus p \to T \big)
        \times
      \sum_{t : T} \Isolated{Q_t}
    \\
    &\simeq
      \sum\nolimits_{(s, p) : \sum_{s : S} \Isolated{P_s}} \sum\nolimits_{(f, t) : (P_s \setminus p \to T \times T)} \Isolated{Q_t}
  \intertext{%
    By \autoref{graft-equiv}, the type \( (P_s \setminus p \to T \times T) \) is equivalent to \( P_s \to T \),
    thus we simplify to
  }
    &\simeq
      \sum\nolimits_{(s, p) : \sum_{s : S} \Isolated{P_s}} \sum\nolimits_{f : P_s \to T} \Isolated{Q_{fp}}
  \intertext{%
    By re-associating the sum yet again, we are left with
  }
    &\simeq
      \sum\nolimits_{(s, f) : \sum_{s : S} (P_s \to T)} \big( \sum_{p : \Isolated{P_s}} \Isolated{(Q_{fp})} \big)
  \end{align*}
  Denote this equivalence by \( \Op{chain}^{\Op{left}}_{F,G} \).
  Now, the left and the right only differ in
  \begin{align*}
    \sum_{p : \Isolated{P_s}} \Isolated{(Q_{fp})}
      \qquad\text{vs.}\qquad
    \Isolated{\big( \sum_{p : P_s} Q_{fp} \big)}
  \end{align*}
  \Autoref{sigma-isolate} gives us a map
  \(
    \SigmaIsolate_{P_s,Q_{f(\shortminus)}} :
      \sum_{p : \Isolated{P_s}} \Isolated{(Q_{fp})}
        \to
      \Isolated{\big( \sum_{p : P_s} Q_{fp} \big)}
  \),
  hence
  \(
    \cramped{
      \Op{chain}_{F,G} \DefEq
        \Op{\Sigma}(\Op{id}, \SigmaIsolate_{P_s,Q_{f(\shortminus)}}) \circ \Op{chain}^{\Op{left}}_{F,G}
    }
  \).
\end{construction}

\begin{theorem}
  For discrete containers \( F, G \), \( \Op{chain}_{F,G} \) is an equivalence of containers.
  \begin{proof}
    Let \( F \JudgeEq (\MkCont{S}{P}), G \JudgeEq (\MkCont{T}{Q}) \).
    By \autoref{discrete-is-equiv-sigma-isolated}
    \( \SigmaIsolate_{P_s,Q_{f(\shortminus)}} \) is an equivalence,
    therefore \( \Op{chain}_{F,G} \) is as well.
  \end{proof}
\end{theorem}

\begin{theorem}
  Assuming a strong chain rule implies that \emph{every} type is discrete.
  Formally,
  \[
    \big( \smashoperator{\prod_{F, G : \Cont}} \Op{isEquiv}(\Op{chain}_{F,G}) \big)
      \to
    \prod_{A : \Type} \Op{Discrete}(A)
  \]
  \begin{proof}
    First observe that for containers \( F \JudgeEq (\MkCont{S}{P}), G \JudgeEq (\MkCont{T}{Q}) \),
    if \( \Op{chain}_{F,G} \) is an equivalence, then so is \( \SigmaIsolate_{P_s,Q_{f(\shortminus)}} \)
    for all \( s : S \) and \( f : P_s \to T \).

    Let \( A : \Type \).
    We use \autoref{is-equiv-sigma-isolate-discrete} to show that \( A \) is discrete.
    Thus, let \( B : A \to \Type \) and prove that \( \SigmaIsolate_{A,B} \) is an equivalence by
    defining containers \( F \) and \( G \) so that the previous observation applies:
    this is the case for \( F \DefEq (\MkCont{1}{A}) \) and \( G \DefEq (\MkCont{A}{B}) \)
    when choosing \( s \DefEq \bullet : 1 \) and \( f \DefEq \Op{id} : A \to A \).
  \end{proof}
\end{theorem}

\begin{corollary}
  Assuming a strong chain rule is inconsistent in the presence of non-set types:
  \[
    \neg \big( \smashoperator{\prod_{F, G : \Cont}} \Op{isEquiv}(\Op{chain}_{F,G}) \big)
  \]
  \begin{proof}
    The circle \( S^1 \) is provably not discrete: If it were, it would be a set.
  \end{proof}
\end{corollary}

\end{document}
