\section{Introduction}

A container \cite{AbbottEtAl2005ContainersConstructingstrictly} is a concise encoding
of an inductive data type:
it consists of a collection of shapes \( S \), each \( s : S \) representing a constructor,
and for each a collection of positions \( P(s) \), indexing arguments to the constructor.
Many operations on data types can be made precise as operations on containers,
including sums \( F \CPlus G \), products \( F \CTimes G \) and substitution \( \Subst{F}{G} \), allowing us to reason about them.
\Citeauthor{Huet1997Zipper} presents \citetitle{Huet1997Zipper}~\cite{Huet1997Zipper},
an informal procedure for computing a type of contexts around a subtree in a tree-like data type.
In~\cite{McBride2001DerivativeRegularType}, \Citeauthor{McBride2001DerivativeRegularType} turns this into an algorithm for the inductively defined class of \enquote{regular types},
and notices that this type of one-hole contexts behaves like a derivative with respect to constants,
sums and products, and that substitution of types follows a chain rule.
This is extended to containers by \citeauthor{AbbottEtAl2003DerivativesContainers},
first in a categorical meta-language~\cite{AbbottEtAl2003DerivativesContainers},
then in a type-theoretic one~\cite{AbbottEtAl2005DataDifferentiating}.
They show that derivatives of containers satisfy a universal property with respect to a class of \enquote{linear}
morphisms of containers,
namely that for any container \( G \) on which equality of positions is decidable,
linear morphisms \( F \multimap \Der{G} \) are in 1-to-1 correspondence with morphisms \( F \CTimes \Id \multimap G \).
Furthermore, the laws of derivatives are encoded as (linear) isomorphisms:
the chain rule, for example, becomes an isomorphism of containers
\( \Der{(\Subst{F}{G})} \cong \Subst{(\Der{F})}{G} \CTimes \Der{G} \).

In the meantime, other notions of containers have appeared,
whose shapes and positions have different kinds of structure:
quotient- \cite{AbbottEtAl2004ConstructingPolymorphicPrograms} and action containers~\cite{JoramVeltri2025DataTypesSymmetries}
express symmetries by having groups act on sets of positions,
symmetric containers internalize symmetries in a groupoid of shapes~\cite{Gylterud2011},
and generalized containers~\cite{AltenkirchKaposi2021containermodeltype} describe directed relations of shapes as a category.

When working in Univalent Foundations, the most direct generalization of containers is that from sets to arbitrary types
with potentially non-trivial higher path types.
What can we say about containers whose shapes and positions are such untruncated types?
Much of the theory transfers directly, since it never assumed that types were sets in the first place.
Even smallest-~\cite{DamatoEtAl2025FormalisingInductiveCoinductive} and largest fixed points~\cite{AhrensEtAl2015NonWellfoundedTrees} can be proved to exist.

In this paper, we show that derivatives can be defined for untruncated containers,
and that they satisfy an analogue of the universal property of discrete containers.
This is mostly an exercise in reverse mathematics:
\emph{What are the minimal assumptions on decidability that we can make such that a derivative is still well-defined?}
To this end, we recall the notion of \emph{isolatedness} of a point in a type, and show how it interacts with various type formers.
Being isolated is a local notion of discreteness, in the sense that around an isolated point, a type is discrete.
Our derivative ranges only over isolated positions;
this allows us to prove the following universal property
by application of simple lemmata for isolated points:
On the (wild) category \( \ContCart \) of containers and cartesian morphisms,
\( \Der \) is an endofunctor that is right-adjoint to taking products, \( \Maybe{\Blank} \).
As a consequence, we can solve a problem left open in \cite[14]{AbbottEtAl2005DataDifferentiating}:
the wild adjunction restricts to an ordinary adjunction on the 1-category of set-truncated containers.
In particular, this proves that \emph{every} set-truncated container has a well-behaved derivative, not just discrete ones.
Our exercise paid off: by generalizing to arbitrary types, we learn something about sets.

Verifying that this generalized derivative satisfies the expected basic laws is straightforward;
we essentially repurpose the proofs for the special case of discrete containers,
but distribute isolated points over type formers whenever necessary.
This works to show that \( \Der \) distributes over sums and products,
but fails for the chain rule:
it generalizes only to a lax chain rule, that is a cartesian morphism
from
\( \Subst{(\Der{F})}{G} \CTimes \Der{G} \)
to
\( \Der{(\Subst{F}{G})} \).

\begin{todo-block}
  Continue here.
\end{todo-block}

\begin{itemize}
  \item
    Two perspectives:
      \begin{itemize}
        \item \emph{(Reverse mathematics)}.
          What are the minimal assumptions on decidability that we can make such
          that a derivative is still well-defined?
        \item \emph{(Topology)}.
          If we think of types as spaces, which points can we remove in a \emph{continuous} manner?
      \end{itemize}
    \item
      Recall notion of isolated points
      \begin{itemize}
        \item not a new notion (\cite{EscardocontributorsTypeTopology}, \cite{KrausEtAl2013GeneralizationsHedberg’sTheorem})
        \item used to define derivatives of arbitrary containers
        \item laws of containers reduce to properties of isolated points
        \item
          Later on (\autoref{lax-chain-rule}),
          it will become necessary to investigate whether isolated points distribute over other type formers,
          in particular dependent sums.
      \end{itemize}
\end{itemize}

\subsection*{Overview}

\begin{enumerate}
  \item Removing points:
    \begin{itemize}
      \item define isolated points: points around which a type is locally trivial
      \item define removal of points, show that it works well for isolated points
      \item define \enquote{grafting} that for functions out of types with a chosen isolated point
    \end{itemize}
  \item Introduce derivatives of untruncated containers
    \begin{itemize}
      \item recall that traditionally, derivatives of containers require discrete shapes,
        or else they won't satisfy a good universal property.
      \item
        define derivatives for all container, but only involving isolated positions
      \item
        yields a (wild) adjunction for arbitrary containers, like in the set-truncated case
      \item
        derivative satisfies the same basic properties with respect to constants,
        sums and products
    \end{itemize}
  \item The chain rule is more involved
    \begin{itemize}
      \item in general, only one direction works, we get a lax chain rule
      \item the lax chain rule is always an embedding
      \item strong iff isolated points distribute over \( \Sigma \)-types
      \item a strong chain rule for arbitrary types is inconsistent; assuming it leads to a contradiction
      \item a strong chain rule for arbitrary sets means that all sets are discrete
    \end{itemize}
\end{enumerate}

\subsection*{Contributions}


\subsection*{Notation}

\begin{itemize}
  \item
    Type formers: unit type \( 1 \),
    empty type \( 0 \),
    products \( A \times B \),
    coproducts \( A + B \),
    function types \( A \to B \)
  \item Dependent type formers:
    dependent function types \( \Pi_{A} B \) or \( \prod_{a : A} B(a) \),
    dependent sum types \( \Sigma_{A} B \) or \( \sum_{a : A} B(A) \)
  \item \( \W \)-types: mention, but introduce later when needed
  \item Map over \( \Sigma \)-types with
    \[
      \Sigma(\Blank, \Blank) : \prod_{f : A \to A'} (\prod_{a : A} B(a) \to B'(f(a))) \to \sum_{A} B \to \sum_{A'} B'
    \]
  \item
    Follow the HoTT-book for transports/substitutions:
    For \( B : A \to \Type \) and \( p : x =_A y \), write \( p_* : B(x) \to B(y) \).
    Not \( \Op{subst} \) or anything like that
  \item
    Truncation levels 
  \item
    Define embeddings by two equivalent properties (invertible \( \Op{cong} \), propositional fibers, propositional fibers over the image)
  \item
    Surjections are maps with merely inhabited fibers
  \item
    Embeddings and surjections form a factorization system for equivalences
  \item equivalences satisfy a 3-for-2 property,
    we apply the insights of \cite{Jong2025FormalizingEquivalencesTears} to minimizes the tears shed over such proofs.
  \item Recall notions of wild category and functor
  \item Use \enquote{filler} for paths of morphism in a wild category
    (since in general, there can be many distinct paths)
  \item wild functor categories are \( [\mathcal{C}, \mathcal{D}] \).
\end{itemize}

