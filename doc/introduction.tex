\section{Introduction}

\begin{itemize}
  \item
    Two perspectives:
      \begin{itemize}
        \item \emph{(Reverse mathematics)}.
          What are the minimal assumptions on decidability that we can make such
          that a derivative is still well-defined?
        \item \emph{(Topology)}.
          If we think of types as spaces, which points can we remove in a \emph{continuous} manner?
      \end{itemize}
    \item
      Recall notion of isolated points
      \begin{itemize}
        \item not a new notion (\cite{EscardocontributorsTypeTopology}, \cite{KrausEtAl2013GeneralizationsHedberg’sTheorem})
        \item used to define derivatives of arbitrary containers
        \item laws of containers reduce to properties of isolated points
        \item
          Later on (\autoref{lax-chain-rule}),
          it will become necessary to investigate whether isolated points distribute over other type formers,
          in particular dependent sums.
      \end{itemize}
\end{itemize}

\subsection*{Notation}

\begin{itemize}
  \item Use \enquote{filler} for paths of morphism in a wild category
    (since in general, there can be many distinct paths)
\end{itemize}

