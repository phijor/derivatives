\section{Introduction}

\begin{itemize}
  \item
    Two perspectives:
      \begin{itemize}
        \item \emph{(Reverse mathematics)}.
          What are the minimal assumptions on decidability that we can make such
          that a derivative is still well-defined?
        \item \emph{(Topology)}.
          If we think of types as spaces, which points can we remove in a \emph{continuous} manner?
      \end{itemize}
    \item
      Recall notion of isolated points
      \begin{itemize}
        \item not a new notion (\cite{EscardocontributorsTypeTopology}, \cite{KrausEtAl2013GeneralizationsHedberg’sTheorem})
        \item used to define derivatives of arbitrary containers
        \item laws of containers reduce to properties of isolated points
        \item
          Later on (\autoref{lax-chain-rule}),
          it will become necessary to investigate whether isolated points distribute over other type formers,
          in particular dependent sums.
      \end{itemize}
\end{itemize}

\begin{remark}
If we restrict the above to set-truncated containers, we can solve a problem left open by \citeauthor{AbbottEtAl2005DataDifferentiating}:
they \enquote{leave it to future work to identify the subcategory of differentiable containers}~\cite[14]{AbbottEtAl2005DataDifferentiating};
indeed, \emph{every} container is differentiable in this sense, as long as we define derivatives in terms of isolated positions:
\end{remark}

\subsection*{Notation}

\begin{itemize}
  \item
    Follow the HoTT-book for transports/substitutions:
    For \( B : A \to \Type \) and \( p : x =_A y \), write \( p_* : B(x) \to B(y) \).
    Not \( \Op{subst} \) or anything like that
  \item Map over \( \Sigma \)-types with
    \[
      \Sigma(\Blank, \Blank) : \prod_{f : A \to A'} (\prod_{a : A} B(a) \to B'(f(a))) \to \sum_{A} B \to \sum_{A'} B'
    \]
  \item
    Define embeddings by two equivalent properties (invertible \( \Op{cong} \) and propositional fibers)
  \item Recall notions of wild category and functor
  \item Use \enquote{filler} for paths of morphism in a wild category
    (since in general, there can be many distinct paths)

  \item Say how to smartly prove that a certain map is an equivalence:
    \cite{Jong2025FormalizingEquivalencesTears} minimizes the tears shed over such proofs.
\end{itemize}

