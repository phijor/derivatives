\section{The Chain Rule}

In addition to the basic properties of the previous section,
we expect the derivative to satisfy an analogue of the chain rule
\( (f \circ g)^{\prime} = (f^{\prime} \circ g) \cdot g^{\prime} \),
which describes how derivatives distribute over composition.
In our setting, substitution \( \Subst{\Blank}{\Blank} \) takes on the role of composition,
and for discrete containers, \citeauthor{AbbottEtAl2005DataDifferentiating} show that
\( \Der{\Subst{F}{G}} \) is indeed isomorphic to \( \Subst{(\Der{F})}{G} \CTimes \Der{G} \).
Na{\"\i}vely attempting to lift their proof to untruncated containers,
we unfortunately run into difficulties:
While it is possible to define a morphism from one to the other,
it is not immediately clear that an inverse exists.
Precisely, we obtain the following \emph{directed} or \emph{lax} chain rule:
\begin{problem}[note={The lax chain rule}]\label{lax-chain-rule}
  For any two containers \( F, G \), define a morphism
  \[
    \Op{chain}_{F,G} :
    \Cart%
      { \Subst{(\Der{F})}{G} \CTimes \Der{G} }%
      {\Der{(\Subst{F}{G})}}
  \]
\end{problem}
\begin{construction}
  Let \( F \JudgeEq (\MkCont{S}{P}) \) and \( G \JudgeEq (\MkCont{T}{Q}) \).
  As usual, we have to construct a map on shapes and an equivalence of positions.
  On shapes, our goal is a map
  \[
    \big(\sum\nolimits_{(s, p) : \sum_{s : S} \Isolated{P_s}} (P_s \setminus p \to T) \big)
      \times
    \sum_{t : T} \Isolated{Q_t}
      \to
    \Big(
      \sum\nolimits_{(s, f) : \sum_{s : S} (P_s \to T)} \Isolated{\big( \sum_{p : P_s} Q_{fp} \big)}
    \Big)
  \]
  Let us first reshape the left side by some equivalences.
  By re-associating the sums, we obtain
  \begin{align*}
      %
    &\mathrel{\hphantom{\simeq}}
      \big(\sum\nolimits_{(s, p) : \sum_{s : S} \Isolated{P_s}} P_s \setminus p \to T \big)
        \times
      \sum_{t : T} \Isolated{Q_t}
    \\
    &\simeq
      \sum\nolimits_{(s, p) : \sum_{s : S} \Isolated{P_s}} \sum\nolimits_{(f, t) : (P_s \setminus p \to T \times T)} \Isolated{Q_t}
    % &\simeq
    %   \adjustlimits\sum_{s : S} \sum_{p : \Isolated{P_s}} \smashoperator[r]{\sum_{(\Blank, t) : (P_s \setminus p \to T \times T)}} \Isolated{Q_t}
  \intertext{%
    The induction principle for \( \Graft \) tells us that the types \( (P_s \setminus p \to T) \times T \) and \( P_s \to T \) are equivalent (\autoref{graft-equiv}),
    thus we simplify to
  }
    &\simeq
      \sum\nolimits_{(s, p) : \sum_{s : S} \Isolated{P_s}} \sum\nolimits_{f : P_s \to T} \Isolated{Q_{fp}}
  \intertext{%
    By permuting the sum yet again, we are left with
  }
    &\simeq
      \sum\nolimits_{(s, f) : \sum_{s : S} (P_s \to T)} \big( \sum_{p : \Isolated{P_s}} \Isolated{(Q_{fp})} \big)
  \end{align*}
  Denote this equivalence by \( \lambda \).
  Now, the left and the right only differ in
  \begin{align*}
    \sum_{p : \Isolated{P_s}} \Isolated{(Q_{fp})}
      \qquad\text{vs.}\qquad
    \Isolated{\big( \sum_{p : P_s} Q_{fp} \big)}
  \end{align*}
  \Autoref{sigma-isolate} gives us a map
  \(
    \SigmaIsolate_{P_s,Q_{f(\Blank)}} :
      \sum_{p : \Isolated{P_s}} \Isolated{(Q_{fp})}
        \to
      \Isolated{\big( \sum_{p : P_s} Q_{fp} \big)}
  \),
  hence
  \(
    \cramped{
      \Op{chain}_{F,G}^{\Sh} \DefEq
        \Op{\Sigma}(\Op{id}, \SigmaIsolate_{P_s,Q_{f(\Blank)}}) \circ \lambda
    }
  \).

  To construct the equivalence on positions,
  let \( s : S \), \( p_0 : \Isolated{P_s} \), \( f : P_s \setminus p_0 \to T \), \( t : T \) and \( q_0 : \Isolated{Q_t} \).
  Our goal becomes to construct an equivalence
  \[
      \big(
        \sum_{p : P_s} Q_{\GraftSyntaxX{f}{t}{p_0}(p)}
      \big) \setminus (p_0 , q_0)
    \simeq
      \big(
        \smashoperator{\sum_{p : P_s \setminus p_0}} Q_{f(p)}
      \big)
        +
      (Q_t \setminus q_0),
  \]
  which we obtain from \autoref{is-equiv-sigma-remove},
  and by applying the computation rules of grafting to \( \GraftSyntaxX{f}{t}{p_0} : P_s \to T \).
\end{construction}

Note that the above proof essentially factors \( \Op{chain}_{F,G} \) into
\[
  \begin{tikzcd}[column sep=large]
    { \Subst{(\Der{F})}{G} \CTimes \Der{G} }%
      \ar[r, -multimap, "\Op{chain}_{F,G}"]
      \ar[d, -multimap, "\sim"{swap}]
      &
    {\Der{(\Subst{F}{G})}}
      \\
    H_0
      \ar[r, -multimap, "\eta"{swap}]
      &
    H_1
      \ar[u, -multimap, "\sim"{swap}]
  \end{tikzcd}
\]
in which
\(
  \eta_{\Sh} :
    { \sum_{s : S} \sum_{f : P_s \to T} \sum_{p : \Isolated{P_s}} \Isolated{Q_{fp}} }%
      \to
    { \sum_{s : S} \sum_{f : P_s \to T} \Isolated{\big( \sum_{p : P_s} Q_{fp} \big)} }
\)
applies \( \SigmaIsolate_{P_s,Q_{f(\Blank)}} \).
We will make this factorization explicit in the derivation of the chain rule for indexed containers (\autoref{binary-chain-rule}).
We say that the chain rule for \( F \) and \( G \) is \emph{strong} if \( \Chain_{F,G} \) is an equivalence of containers;
the above lets us immediately see when under which circumstances we can expect this to be the case.
\begin{proposition}\label{strong-chain-rule-iff-is-equiv-sigma-isolate}
  Let \( F = (\MkCont{S}{P}) \) and \( G = (\MkCont{T}{Q}) \).
  The following are equivalent propositions:
  \begin{enumerate}
    \item
      \( \Op{chain}_{F,G} \) is an equivalence of containers
    \item
      \( \SigmaIsolate_{P_s,Q_{f(\Blank)}} \) is an equivalence for all \( s : S \) and \( f : P_s \to T \)
  \end{enumerate}
  \begin{proof}
    By inspection of the definition of \( \Op{chain} \) in terms of \( \SigmaIsolate \).
  \end{proof}
\end{proposition}
We can use any of the equivalent properties listed in \autoref{is-equiv-sigma-isolate-iff-isolated-pair}
to express when exactly we have a strong chain rule.
In particular, we know that \( \SigmaIsolate \) is always an embedding (cf.~\autoref{is-embedding-sigma-isolate}),
hence we can think of \( \Chain_{F,G} \) as embedding \( \Subst{(\Der{F})}{G} \CTimes \Der{G} \)
as a sub-container inside of \( \Der(\Subst{F}{G}) \).

As expected by \cite[{Proposition~6.6}]{AbbottEtAl2005DataDifferentiating},
the chain rule is strong for \emph{discrete} containers:
\begin{theorem}\label{discrete-strong-chain-rule}
  For discrete containers \( F, G \), \( \Op{chain}_{F,G} \) is an equivalence.
  In particular, it is an isomorphism in the 1-category of set-truncated containers.
  \begin{proof}
    Positions of \( F \) and \( G \) are discrete,
    so by \autoref{discrete-is-equiv-sigma-isolated}
    \( \SigmaIsolate \) is an equivalence.
    By \autoref{strong-chain-rule-iff-is-equiv-sigma-isolate}, \( \Op{chain}_{F,G} \) is an equivalence of containers.
  \end{proof}
\end{theorem}

Secondly, we conclude that globally having a strong chain rule is an inherently classical property:
\begin{theorem}\label{globally-discrete-iff-strong-chain-rule}
  The following are equivalent propositions:
  \begin{enumerate}
    \item \emph{every} type is discrete
    \item for all containers \( F \) and \( G\), \( \Op{chain}_{F,G} \) is an equivalence
  \end{enumerate}
  \begin{proof}
    If every type is discrete, then so is every container, hence the chain rule is always an equivalence by \autoref{discrete-strong-chain-rule}.
    In the other direction,
    use \autoref{discrete-iff-is-equiv-singl-isolate} to show that any given type \( A \) is discrete
    ---
    that is, given some \( a_0 : A \), prove that \( \SigmaIsolate_{A, {a_0 = \Blank}} \) is an equivalence.
    We do so by applying \autoref{strong-chain-rule-iff-is-equiv-sigma-isolate} to containers
    \( F \DefEq (\MkCont{1}{A}) \) and \( G \DefEq (\MkCont{(a : A)}{a_0 = a}) \).
  \end{proof}
\end{theorem}

As a consequence, globally assuming that the chain rule is strong is inconsistent in the presence of types of higher truncation level:
The circle \( S^1 \) is provably not discrete (if it were, it would be a set!),
hence
\(
  \neg ( \prod_{F, G : \Cont} \Op{isEquiv}(\Op{chain}_{F,G}) )
\).

If instead we restrict ourselves to the world of sets,
then we can conclude that a globally strong chain rule exists if and only if arbitrary equalities are decidable:
\begin{corollary}
  The following are equivalent:
  \begin{enumerate}
    \item Every set is discrete.
    \item In the 1-category of set-truncated containers, \( \Chain_{F,G} \) is an isomorphism
      for all pairs of containers \( F \) and \( G \).
  \end{enumerate}
  \begin{proof}
    In the 1-category \( \ContCart_{0,0} \), being an isomorphism is a proposition,
    which is equivalent to being an equivalence of containers.
    The claim follows by inspection of the proof of \autoref{globally-discrete-iff-strong-chain-rule},
    and ensuring that the same argument applies even when all types involved are sets.
  \end{proof}
\end{corollary}

We have seen that our definition of derivative behaves nicely even in the presence of non-discrete types,
in that we retain the ways it interacts with sums and products (\autoref{sum-product-rule}).
Its interaction with substitution, however, is more subtle.
While we do obtain a chain rule, it is now \emph{directed} or \emph{lax} (\autoref{lax-chain-rule}),
and whether we can strengthen it to an equivalence depends on the pair of containers involved (\autoref{strong-chain-rule-iff-is-equiv-sigma-isolate}).
Indeed, assuming the latter for any pair of containers is inconsistent in the presence of higher inductive types such the circle \( S^1 \).
