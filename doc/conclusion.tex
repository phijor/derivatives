\section{Conclusion}

We have seen that extending a derivative operation to \( \Type \)-valued containers is possible:
If the obstruction to a general derivative is that positions are not necessarily discrete,
simply restrict a derivative to the subtype of those positions that are!
This generalization does not come for free, however.
Although it is universal and retains basic properties, the chain rule is no longer in general invertible,
and neither is the \( \mu \)-rule.
In fact, a global chain rule is impossible, and at least constructively taboo when restricted to sets.
Interestingly, the status of the \( \mu \)-rule is somewhat weaker:
In case of the chain rule the contradiction arises because we are free to apply it to an arbitrary pair of containers.
In \autoref{strong-mu-rule-iff-strong-chain-rule}, however, the constraints are tighter
---
there is one degree of freedom (the container \( F \)),
and whether the rule is strong depends on that and \( \mu F \),
which is canonically derived from \( F \).
We would like to know:
Is a strong \( \mu \)-rule a constructive taboo in the same way that a strong chain rule is?

In the future we would also like to work out the behaviour of \( \Der \) on fixed points other than \( \mu \).
In their original work~\cite{AbbottEtAl2005DataDifferentiating}, \citeauthor{AbbottEtAl2005DataDifferentiating} derive a fixed-point rule also for the largest fixed point, \( \nu F \).
Recently, \citeauthor{DamatoEtAl2025FormalisingInductiveCoinductive} show that \( \Type \)-valued container functors preserve both smallest and largest fixed points~\cite{DamatoEtAl2025FormalisingInductiveCoinductive}.
In either work, many of the intermediate lemmata hold for arbitrary fixed points,
and the same is true for a number of our results:
The construction of the lax chain rule in \autoref{lax-chain-rule}, for example, only depends on the property of \( \In_F : \Cart{\Subst{F}{\mu F}}{\mu F} \) being \emph{some}
\( \Subst{F}{\Blank} \)-fixpoint, not necessarily the smallest.
We could thus give for any \( \varphi : \CartIso{\Subst{F}{\varphi F}}{\varphi F} \) a \enquote{\( \varphi \)-rule}
as an embedding \( \operatorname{\Op{\varphi-rule}} : \Cart{\mu F'}{\Der(\varphi F)} \) in which
\(
  F' \DefEq
    { \Wk{\Der_0{F}[ {\varphi F} ]} }
      \CPlus
    (
      \Wk{\Der_1{F}[ {\varphi F} ]}
        \CTimes
      \Proj{1}
    )
\).

In another direction, we are curious whether we can describe more than just one-hole contexts.
\Citeauthor{AbbottEtAl2003DerivativesContainers} define \emph{linear exponentials} \cite[{Def.~4.1}]{AbbottEtAl2003DerivativesContainers}:
for containers \( F \) and \( H \), the linear exponential of \( [H, F] \) exists if there is a universal morphism from \( \Blank \CTimes H \) to \( F \).
They show that \( \Der{F} = [\Id, F] \) if \( F \) is discrete.
The adjunction of \autoref{derivative-adjunction-sets} is such a universal morphism, natural in all \( F \),
hence \( \Der = [\Id, \Blank] \) as functors.
Iterating the adjunction, we can describe \( n \)-hole contexts as linear exponentials \( \Der^n = [\Op{y}\Fin{n}, \Blank] \).
We conjecture that \( [H, \Blank] \) exists for set-truncated \( H \) with finitary positions,
since any such \( H \) is \enquote{just} a big coproduct of representables \( \Op{y}\Fin{n_i} \),
This would give us a type of contexts with \enquote{\( H \)-shaped} holes.

% In \cite{AbbottEtAl2003DerivativesContainers}, \citeauthor{AbbottEtAl2003DerivativesContainers} consider the general adjunction \( \Blank \CTimes H \dashv [H, \Blank] \),
% perhaps we can make more progress towards that? By iterating ours, it should be possible to derive \( \Blank \CTimes \Op{K}(n) \dashv \Der^n \).
% For combinatorial species, this is enough to derive the fully general adjunction, since (1) the \( \Op{K}(n) \) are exactly the representables
% \( \Op{y}(n) \), (2) every species is a colimit of some \( \Op{y}(n_i) \)'s, and (3) the product (in this case, Day-convolution) is co-continuous.
